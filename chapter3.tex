\chapter{Implementation}
\label{cha:implementation}

In this chapter we discuss the reasoning behind the implementation choices we made during the development of the program. We start by defining the syntax and semantics of an SSA form intermediate language and then we proceed to the various steps for the register allocation, namely liveness analysis, coloring and SSA destruction.

\section{Syntax}
In creating the syntax of our intermediate language the main objective is to define types that make it hard to represet illegal SSA programs as we described in \Cref{sec:ssa}. We start by defining the registers.

\subsection{Registers}
During the first part of the register allocation our representation will make use virtual registers and, in the second part we will switch to a version of the language that uses physical registers. To do this we need to define a generic implementation of the language that can work with both virtual and physical registers.
Fortunately, Coq allows us to use type classes to achieve this. We can define a class that represents a generic register, whose implementation can be easily swapped as needed.

\begin{lstlisting}[language=Coq]
Class RegClass := {
  reg : Set;
  reg_eqb : reg -> reg -> bool;
  reg_eq_dec : forall r r' : reg, {r = r'} + {r <> r'};
}.
\end{lstlisting}

For every Coq module we need to provide an instance of this class that defines \textbf{reg}, the set of all registers, \textbf{reg\_eqb}, a boolean function to check for equality between registers and \textbf{reg\_eq\_dec}, a proof for the decidability of the equality relation between two registers.

For virtual registers, we want to use a set of infinite size, we can simply do this by defining the set of registers to be the set of natural numbers, even though in Coq this set is bound by the stack size we can assume it can represent an infinite amount of registers.
For physical registers instead we use a custom set that contains the phsyical registers of an already existing assembly language, for the sake of this project we chose the 64 bit registers of the x86 architecture.
For now on every time we mention a register we must associate it to the virtual or physical implementation depending on the context.

\subsection{Normal Instructions}

Like many intermediate languages, ours will include some basic typing, in particular, a value in our language could be an immediate value implemented as an integer, a register or a pointer to a memory location implemented as a natural. The value type is defined as follows:

\begin{lstlisting}[language=Coq]
Definition ptr : Type := nat.

Inductive val : Type :=
  | Imm (x : Z)
  | Reg (r : reg)
  | Ptr (p : ptr)
.
\end{lstlisting}

The next step is to define the expression type, which will include the operations supported by our language, in particular we define operations for copying a value from a register, getting the content from a memory location and performing logic and arithmetic calculations. The resulting type is defined as follows:

\begin{lstlisting}[language=Coq]
Inductive expr : Type :=
  | Val : val -> expr
  | Load : val -> expr
  | Add : reg -> val -> expr
  | Sub : reg -> val -> expr
  ...
.
\end{lstlisting}

The first operator of a binary expression is of type \textbf{reg} because we assume that an expression cannot have two constant arguments, since in that case the result could be calculated during a previous step of constant folding.
Another thing to note is that, for the sake of register allocation, a differentiation between unary, binary and n-ary expressions is useless as we are only concerned with the operators of an expression (the registers) and not the operation.

Now it's finally time to define our instruction type, which is a rather simple task:

\begin{lstlisting}[language=Coq]
  Inductive inst : Type :=
  | Def (r : reg) (e : expr)
  | Store (v : val) (r : reg)
.
\end{lstlisting}

As we can see, we also define a constructor for the store operation, used for storing the content of a register into a memory location, we do this here because the store is the only operation that doesn't produce a value as a result and so it cannot be represented as an assignment of an expression to a register.

\subsection{Phi Instructions}
As was shown in \Cref{sec:ssa} a phi instruction is an assignment to a register based on the previous block in the control flow, it can then be represented as follows:

\begin{lstlisting}[language=Coq]
Definition phi_arg : Type := (reg * lbl).

Inductive phi : Type :=
  | Phi (r : reg) (rs: list phi_arg)
.
\end{lstlisting}

As we can see at line 1 we use the \textbf{lbl} type, which represents a block label. In the beginning of the project we tried to opt for an intermediate representation that does not use labels but rather uses direct links to the beginning of blocks. But, as we will see later, this introduces some difficulties in the definition of the sematics of the language.

\subsection{Blocks and Jump Instructions}

The reason why we defined phi instructions and normal instructions separately is that we can then separate the two sections in the definition of blocks, making it mandatory to have an initial section of phi instructions and then a second section of normal instructions, finally we define the jump instruction type that we must put at the end of a block. At the start of a block we must also specify its label.

\begin{lstlisting}[language=Coq]
CoInductive block : Type :=
  | Block (l : lbl) (ps : list phi) (is : list inst) (j : jinst)

with jinst : Type :=
  | CondJump : cond -> reg -> val -> block -> block -> jinst
  | Jump : block -> jinst
  | Halt : jinst
.
\end{lstlisting}

The jump instruction points directly to the successor block(s), having to provide the successor blocks instead of their labels avoids jumping to unexistent blocks.
We can easily see that the previous one is the definition of a control flow graph where each block is a node and the jump instructions defines an edge relation between predecessor and successor blocks.

Finally since a CFG (which we call program) as we described in \Cref{subsec:cfg} is a set of blocks where we identify a start block and a control flow relation, it is sufficient to provide this definition

\begin{lstlisting}[language=Coq]
Definition program : Type := block.
\end{lstlisting} as the first block of the CFG is the first block we encounter in a visit, the set of blocks is the set of blocks reachable in a complete visit of the graph and the control flow relation is the edge relation of the graph.

\subsection{Limitations}
The intermediate language we just defined still has some problems that we cannot solve using the Coq type system. The following are some examples of ill-formed programs that are still legal in our representation, we also introduced some notations to make our language more readable.
The first two rules we stated in \Cref{sec:ssa} are not enforced, we can have multiple assignemnts to the same variable and we can use variables that are not defined as is shown in the following definitions:

\begin{lstlisting}[language=Coq]
Definition double_assignment : block :=
  Block 0 [] [
    r(1) <- (Imm 0);
    r(1) <- (Imm 1)
  ]
  Halt
.

Definition undefined_variable : block :=
  Block 0 [] [r(1) <- (Reg 0)] Halt
.
\end{lstlisting}

The introduction of labels makes it possible to have multiple blocks with the same label. Here is an example:

\begin{lstlisting}[language=Coq]
Definition double_lbl_2 : block := Block 0 [] [] Halt.
Definition double_lbl_1 : block := Block 0 [] [] (Jump double_lbl_2).
\end{lstlisting}

Finally phi instructions are prone to containing inconsistent data. The problem is that the arguments of a phi instruction may not necessarily match the predecessors of the block. We show an example here:

\begin{lstlisting}[language=Coq]
Definition ill_formed_phi_2 : block :=
  Block 1 [r(0) <- phi [(1, 0); (2, 1); (3, 2)]] [] Halt
.
Definition ill_formed_phi_1 : block :=
  Block 0 [] [] (Jump ill_formed_phi_2)
.
\end{lstlisting}


\section{Semantics}
To help us reason about the semantics of the language we create interpreter.

\section{Liveness Analysis}
\section{Coloring}
\section{SSA Destruction}