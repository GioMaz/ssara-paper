\documentclass[11 pt,a4paper,twoside]{book}

% General

\usepackage[utf8]{inputenc}
\usepackage{syntax}
\usepackage{geometry}
\usepackage{csquotes}
\usepackage{graphicx}
\usepackage{listings}
\usepackage[svgnames]{xcolor}
\usepackage{comment}
\usepackage{setspace}
\usepackage{appendix}
\usepackage{lipsum}     %testo di prova (si può cancellare una volta iniziato a scrivere)
\usepackage{lineno}
% \linenumbers

%\usepackage[9-15]{pagesel}
% attenzione: la filigrana in prima pagina ha opacità massima con questa opzione attivata
% non mettendo la prima pagina comunque appare draft in background

% Math
\usepackage{amsmath}
\usepackage{amssymb}
\usepackage{physics}
\usepackage{braket}
\usepackage{mathpartir}
\usepackage{bussproofs}

% Front page
\usepackage{frontespizio}
\usepackage{tikz}
\usepackage[pages=some]{background}

% Code
\usepackage{alltt}
\usepackage{listings}
\lstset{
    basicstyle=\ttfamily,  % Same font as alltt
    columns=fullflexible,  % Match spacing of alltt
    keepspaces=true        % Preserve spaces like alltt
}

\definecolor{myblue}{RGB}{0,0,180}
\definecolor{mygreen}{RGB}{0,120,0}
\definecolor{myred}{RGB}{180,0,0}
\definecolor{mypurple}{RGB}{120,0,120}

\lstdefinestyle{C}{
    language=C,
    mathescape=true,
    tabsize=4,
    keywordstyle=\color{myblue},              % Keywords in blue
    commentstyle=\color{mygreen!50!black},    % Comments in green
    stringstyle=\color{orange},             % Strings in orange
    showstringspaces=false,
    numbers=left,
    numberstyle=\tiny\color{gray},
    stepnumber=1,
    % frame=single,
    breaklines=true
}

\lstdefinelanguage{Rocq}{
    morekeywords=[1]{Definition, Inductive, CoInductive, Class, Instance, Fixpoint, CoFixpoint, Function, Proof, Qed, Module},
    morekeywords=[2]{forall, exists, match, with, end, if, then, else, let, in, fix, fun, intros, unfold, destruct, lia},
    morekeywords=[3]{Prop, Type, Set, True, False, nat, bool, option, list},
    sensitive=true,
    morecomment=[l]{(*},
    morecomment=[s]{(*}{*)},
    morestring=[b]",
}

\lstdefinestyle{Rocq}{
    language=Rocq,
    mathescape=true,
    keywordstyle=[1]\color{myblue},
    keywordstyle=[2]\color{mypurple},
    keywordstyle=[3]\color{myred},
    tabsize=2,
    commentstyle=\color{green!50!black}\itshape,
    stringstyle=\color{orange},
    showstringspaces=false,
    numbers=left,
    numberstyle=\tiny\color{gray},
    stepnumber=1,
    % frame=single,
    breaklines=true
}

% OCaml language with keyword groups
\lstdefinelanguage{OCaml}{
  sensitive=true,
  morecomment=[s]{(*}{*)},
  morestring=[b]",
  morekeywords=[1]{
    let, rec, in, and, val, type, module, struct, sig, class, object, method,
    inherit, virtual, functor, include, open, external, exception
  },
  morekeywords=[2]{
    if, then, else, match, with, try, when, for, to, downto, while, do, done,
    begin, end, fun, function
  },
  morekeywords=[3]{true, false},
  morekeywords=[4]{of, mutable, private, new, lazy, as, assert}
}

\lstdefinestyle{OCaml}{
    language=OCaml,
    mathescape=true,
    keywordstyle=[1]\color{myblue}\bfseries,    % Declarations
    keywordstyle=[2]\color{mypurple}\bfseries,  % Control flow
    keywordstyle=[3]\color{orange}\bfseries,    % Constants
    keywordstyle=[4]\color{teal}\bfseries,      % Misc keywords
    tabsize=2,
    basicstyle=\ttfamily\small,
    commentstyle=\color{green!50!black}\itshape,
    stringstyle=\color{red},
    showstringspaces=false,
    numbers=left,
    numberstyle=\tiny\color{gray},
    stepnumber=1,
    % frame=single,
    breaklines=true,
}

\lstdefinelanguage{NASM}{
  morekeywords={
    mov, add, sub, mul, div, xor, and, or, inc, dec, jmp, cmp, je, jne, jl, jle, jg, jge, call, ret, push, pop, syscall
  },
  sensitive=true,
  morecomment=[n]{;}{}, % NASM uses ';' for comments
  morestring=[b]",      % Strings in double quotes
  morestring=[b]',      % Strings in single quotes
}

\lstdefinestyle{NASM}{
  language=NASM,
    keywordstyle=\color{myblue}\bfseries,
  numbers=left,
  numberstyle=\tiny\color{gray},
  stepnumber=1,
  % frame=single,
  basicstyle=\ttfamily,
  breaklines=true
}

% Page header
\usepackage{fancyhdr}
\pagestyle{fancy}
\fancyhf{}
\fancyhead[LE]{\leftmark}
\fancyhead[RO]{\rightmark}
\fancyfoot[LE,RO]{\thepage}


% Bibliography
\usepackage[style=numeric]{biblatex}
\addbibresource{library.bib}

% Abstract
\newenvironment{abstract}%
    {\cleardoublepage%
        \thispagestyle{empty}%
        \null \vfill\begin{center}%
            \bfseries \abstractname \end{center}}%
        {\vfill\null}

% Index
\usepackage{imakeidx}
\makeindex

% Symbols and notations
\usepackage{nomencl}
\makenomenclature

% To load as last packages to avoid problems
\usepackage{cleveref}
\usepackage{glossaries}
\makeglossaries
\newacronym{ra}{RA}{Register Assignment}
\newacronym{jair}{JAIR}{Just Another Intermediate Representation}
\newacronym{ssa}{SSA}{Static Single Assignment}
\newacronym{peo}{PEO}{Perfect Elimination Order}
\newacronym{cfg}{CFG}{Control Flow Graph}
\newacronym{bnf}{BNF}{Backus-Naur Form}

\usepackage{graphicx} % Required for inserting images
\usepackage[T1]{fontenc}
\usepackage[utf8]{inputenc}
% \usepackage[italian]{babel}
\usepackage{amsmath}
\usepackage{dsfont}
\usepackage{amssymb}
\usepackage{amsthm}
\usepackage{array}
\usepackage{physics}
% \usepackage[table]{xcolor} %caricalo prima del nicematrix
\usepackage{nicematrix}
\newcolumntype{S}[1]{>{\raggedright\arraybackslash}m{#1}}
\newcolumntype{D}[1]{>{\raggedleft\arraybackslash}m{#1}}
\newcolumntype{M}[1]{>{\centering\arraybackslash}m{#1}}
\usepackage[all]{xy}
\newcolumntype{L}{>{$}l<{$}}
\newcolumntype{C}{>{$}c<{$}}
\newcolumntype{R}{>{$}r<{$}}
\usepackage{csquotes}
\renewcommand{\arraystretch}{2}
\usepackage{bbm}
\DeclareMathOperator*{\argmax}{argmax}
\DeclareMathOperator*{\argmin}{argmin}
\newcommand{\prob}[1]{\mathbb{P} \left( #1 \right )}
\newcommand{\E}[1]{\mathbb{E} \left[ #1 \right]}
\newcommand{\EC}[2]{\mathbb{E} \left[ #1 | #2\right]}
\newcommand{\betaold}{\beta^{\text{old}}}
\newcommand{\sigmamat}{\mathbf{\Sigma}}
\newcommand{\addvertex}[3]{#1 \oplus_{#2} #3}
\usepackage{todonotes}
\usepackage{imakeidx}
\usepackage{subcaption}
\usetikzlibrary{arrows.meta, positioning}
\setlength{\parindent}{0pt}
\newcommand{\nt}[1]{\langle\text{#1}\rangle}

\makeindex

\begin{document}

\theoremstyle{definition}
\newtheorem{Def}{Definizione}[section]

\theoremstyle{remark}
\newtheorem{pos}{Postulato}[section]

\theoremstyle{definition}
\newtheorem{prop}{Proposizione}[section]

\theoremstyle{definition}
\newtheorem{oss}{Osservazione}[section]

\theoremstyle{plain}
\newtheorem{teo}{Teorema}[section]

\theoremstyle{definition}
\newtheorem{metodo}{Metodo}


\Universita{University of Trento}
\Dipartimento{Department of Information Engineering and Computer Science}
\CorsoDiLaurea{Bachelor's Degree in Computer Science}
\AnnoAccademico{Academic Year 2024--2025}
\Titolo{Towards a Verified Implementation of SSA-Based Register Assignment}
\Relatore{Prof. Marco \textsc{Patrignani}}
\Correlatore{Matthis \textsc{KRUSE}}
\RelatoreLabel{Supervisor}
\CorrelatoreLabel{Co-Supervisor}
\CandidatoLabel{Student}

\Candidato{Giovanni Maria \textsc{Zanchetta}} 
\Matricola{236196}
\DataEsame{September 2025}
\Logo{layout/logo-red.png}
\LogoWidth{4.5cm} %optional, default: 3cm
\LogoPosition{top}
\LogoSfondo{layout/logo-grayscale.png}
\opacitaSfondo{0.05}

\begin{titlepage}
    \newgeometry{left=3cm, right=3cm, bottom=2cm, top =3cm} 
    \pagestyle{empty}
    \makefrontpage
    \restoregeometry
\end{titlepage}

\frontmatter

% Acknowledgments
\chapter*{Acknowledgments}
% Error

First, I would like to express my gratitude to my supervisor, Prof. Patrignani. Then, I would like to thank my co-advisor Matthis Kruse, who has been extremely available during the past few months, and has offered me great help with the project, often steering my attention to the right problems to address, and helping me learn many things.

Additionally, I would like to thank my parents, Morena and Stefano, who supported me throughout this journey both financially and emotionally, helped me make important decisions along the way, and always encouraged me to pursue my academic goals.

Another acknowledgment goes to my siblings, Maddalena and Simone, whose company made my frequent returns home more enjoyable.

Finally, I want to thank my friends and colleagues from the San Bartolameo dorms and from my course, with whom I had the pleasure of sharing most of my time over the last three years.

\tableofcontents

% \printglossaries
% \addcontentsline{toc}{chapter}{Glossario}

% \printnomenclature
% \addcontentsline{toc}{chapter}{Lista della Nomenclatura}

\mainmatter

% Introduction
% Errors

\chapter{Introduction}
\label{cha:intro}

While high-level programming languages such as C and OCaml allow developers to define an arbitrary number of variables, actual hardware imposes strict limits. For example, the x86 architecture provides only 8 general-purpose registers \cite{intel_sdm_vol1}.
In order to handle this limitation, a register allocation pass is required. Register allocation consists of different steps, namely spilling, register assignment, and coalescing. For this project we focus on register assignment, which is a crucial part of the pipeline.

To clearly state the purpose of register assignment, take into consideration the following example: Suppose we have the C function shown in \Cref{fig:cbefore}, and we want to compile it for an architecture with three register. The non-trivial goal of register assignment is to identify which physical register needs to be used to store a new variable whenever the total number of variables is greater than the number of physical registers. The resulting program after register assignment is shown in \Cref{fig:cafter}. In particular, we see that at line 5, the register $r_2$ is reused, as its content is no longer needed.

\begin{minipage}{0.48\linewidth}
\centering
\lstset{style=C}
\begin{lstlisting}[caption={C program returning $2a+b$.}, label={fig:cbefore}]
int foo() {
    int a = get_a();
    int b = get_b();
    int c = a + b;
    int d = a + c;
    return d;
}
\end{lstlisting}
\end{minipage}
\hfill
\begin{minipage}{0.48\linewidth}
\centering
\lstset{style=C}
\begin{lstlisting}[caption={Same C program after register assignment.}, label={fig:cafter}]
int foo() {
    int $r_1$ = get_a();
    int $r_2$ = get_b();
    int $r_3$ = $r_1$ + $r_2$;
    $r_2$ = $r_1$ + $r_3$;
    return $r_2$;
}
\end{lstlisting}
\end{minipage}

Any compiler has at least a basic implementation of register assignment, some even employ a verified implementation of this phase. The CompCert \cite{Rideau-Leroy-regalloc} verified compiler, for example, does not actually verify register assignment, instead, it verifies a separate checker that validates the output of the algorithm. CakeML \cite{10.1145/2578855.2535841}, instead, implements fully verified linear-scan register assignment which, despite having linear time complexity, is a greedy algorithm and may produce suboptimal assignments.
For arbitrary programs, register assignment is NP-complete. However, this problem is solvable optimally and in polynomial time for programs in \gls{ssa} form.
To the best of our knowledge, no mechanized implementation of SSA-based register assignment currently exists. To this end, we make the following contributions:

\begin{itemize}
    \item We introduce JAIR (Just Another Intermediate Representation), an intermediate representation in SSA form that takes into consideration the stages of the register assignment pipeline, from liveness analysis to SSA destruction. In \Cref{sec:jair}, we define the syntax of JAIR and discuss the feasibility of making illegal SSA-form programs unrepresentable.

    \item \Cref{sec:jair-int} defines the dynamic semantics of JAIR as a virtual machine that simulates the execution of JAIR programs;

    \item In \Cref{sec:liveness}, \Cref{sec:ra}, and \Cref{sec:destruct} we implement a register assignment algorithm for JAIR programs in Rocq;

    \item In \Cref{sec:extract}, we extract the whole pipeline as OCaml and bundle it into a single procedure, namely \texttt{regassign}. After that, we compare the output of programs executed on the previously defined virtual machine \textit{before} register assignment, with the output of the same programs run on an x86-64 machine \textit{after} register assignment;

    \item In \Cref{cha:verification} we discuss the verification of our register assignment pass. That is, we prove the termination and the partial correctness;
\end{itemize}

In the remainder of the thesis, we discuss related work in \Cref{cha:relwork}, give some general background in \Cref{cha:background}, and finally conclude in \Cref{cha:conclusions}.

% At a high level, the process begins with liveness analysis, then we proceed with register allocation and end with SSA destruction. The whole pipeline is developed using the Rocq Proof Assistant, which allows us to verify the correctness of the core register assignment algorithm. Once verified, the code is then extracted into an efficient language, OCaml, and compiled into a working binary.

% Body
% Errors

\chapter{Background}
\label{cha:background}

In this chapter, we present the foundational tools and concepts employed throughout this project. We begin by introducing the Coq Proof Assistant, highlighting the features that make it particularly suitable for our implementation. We then describe the Single Static Assignment (SSA) form, the intermediate representation used in our compiler, along with the necessary structural elements to make it practical. Finally, we discuss the theoretical foundations of register allocation, with particular attention to its formulation within the SSA paradigm.

\section{The Coq Proof Assistant}

The core of our register allocator is implemented in \textit{Coq}, an interactive proof assistant based on a dependently typed functional language known as \textit{Gallina}. While Coq is primarily used for formal verification, in this project we focus more on its computational capabilities than on the formal proofs of correctness.

\subsection{Function Termination}
\label{subsec:funterm}

A fundamental feature of Coq is its enforcement of termination in all function definitions. Recursive functions are typically defined using the \texttt{Fixpoint} keyword, which requires pattern matching over inductive types. Coq statically checks that, for each recursive call, there exists a structurally decreasing argument, guaranteeing termination by eventually reaching a base case.

Some recursive functions, however, do not conform to this syntactic criterion. In such cases, Coq offers two common workarounds:

\begin{enumerate}
    \item \textbf{Fuel-based recursion}: This approach involves adding an artificial decreasing argument (commonly named \texttt{fuel} of type \texttt{nat}) to the function signature. The function then proceeds recursively while decrementing \texttt{fuel} at each step. Termination is guaranteed when \texttt{fuel} reaches zero. The drawback is that an upper bound on the number of iterations must be known in advance, which may be difficult to determine and, if underestimated, can lead to incomplete computations.
    
    \item \textbf{Using \texttt{Function} with a termination proof}: This alternative allows defining more general recursive functions while providing an explicit proof of their termination. Although more flexible, this method introduces additional proof obligations and complexity.
\end{enumerate}

In this project, we primarily adopt the fuel-based approach, assuming that appropriate bounds on the number of iterations are known for the algorithms in question.

\subsection{OCaml Extraction}
% Content to be added.

\section{Single Static Assignment (SSA)}
\label{sec:ssa}

We adopt the Single Static Assignment (SSA) form as our intermediate representation, due to its structural properties that greatly simplify register allocation. An SSA-form program satisfies the following constraints:

\begin{itemize}
    \item Each variable (or virtual register) is assigned exactly once;
    \item Every use of a variable is dominated by its unique definition.
\end{itemize}

A label $l_1$ is said to \textit{dominate} a label $l_2$ if every path from the entry point of the control flow graph (CFG) to $l_2$ passes through $l_1$.

To make SSA a practical intermediate representation, we must introduce additional structures, namely the control flow graph and $\phi$-functions.

\subsection{Control Flow Graphs}
\label{subsec:cfg}

To represent the control flow of programs, we employ Control Flow Graphs (CFGs). A CFG is defined as the triple $(B, CF, \textbf{start})$, where:

\begin{itemize}
    \item $B$ is the set of \textit{basic blocks}, each consisting of a sequence of instructions preceded by a $\phi$ section and terminated by a control transfer instruction (e.g., branch or jump).
    \item $CF \subseteq B \times B$ is the control flow relation, indicating allowed transitions between blocks.
    \item $\textbf{start} \in B$ is the designated entry point of the program.
\end{itemize}

\subsection{Phi Operations}
\label{subsec:phi}

% Due to the single-assignment constraint in SSA, merging variable values from multiple control paths requires the use of \textit{$\phi$-functions}. A $\phi$-instruction has the general form:

% \[
% x \leftarrow \phi((y_1, b_1), (y_2, b_2), \dots, (y_n, b_n))
% \]

% Here, each pair $(y_i, b_i)$ indicates that the value of $y_i$ should be assigned to $x$ if the program arrives from block $b_i$. $\phi$-functions are always placed at the beginning of a basic block, and are semantically executed during the control transfer from a predecessor block.

% We often group multiple $\phi$-assignments into a matrix-like section:

% \[
% \begin{pmatrix}
%     x_1 \\ x_2 \\ \vdots \\ x_m
% \end{pmatrix}
% \leftarrow \phi
% \begin{pmatrix}
%     (y_{11}, b_1) & (y_{12}, b_2) & \dots & (y_{1n}, b_n) \\
%     (y_{21}, b_1) & (y_{22}, b_2) & \dots & (y_{2n}, b_n) \\
%     \vdots & \vdots & \ddots & \vdots \\
%     (y_{m1}, b_1) & (y_{m2}, b_2) & \dots & (y_{mn}, b_n) \\
% \end{pmatrix}
% \]

% An important semantic property is that all $\phi$-assignments in a block are executed \textit{in parallel}. This aspect must be carefully preserved when translating SSA code to machine code.

\section{Register Allocation}
\label{sec:ra}

Register allocation is the process of assigning a potentially unbounded number of program variables to a limited number of physical registers. A well-known approach reduces this problem to graph coloring.

Given a program, we construct its \textit{interference graph} $G = (V, E)$, where:

\begin{itemize}
    \item $V$ is the set of variables (or virtual registers);
    \item $(u, v) \in E$ if variables $u$ and $v$ are simultaneously live at some program point, and therefore cannot share the same register.
\end{itemize}

A valid $k$-coloring of the interference graph, where each color represents a physical register, corresponds to a feasible register assignment for a machine with $k$ registers.

\subsection{Register Allocation in SSA Form}

While graph coloring is $\mathcal{NP}$-complete for arbitrary graphs, the interference graphs of programs in SSA form have a special structure: they are \textit{chordal} graphs. A chordal graph is one in which every cycle of four or more vertices has a chord, i.e., an edge connecting two non-consecutive vertices.

This structural property allows optimal register allocation to be performed in polynomial time, specifically, in quadratic time, making SSA form not only more analyzable but also more efficient for code generation tasks such as register allocation.


% Errors

\chapter{\gls{jair}}
\label{cha:jair}

Here, we detail the implementation of \gls{jair}. In particular, in \Cref{sec:jair-syntax}, we define the syntax of the language in \gls{bnf}, and we show its implementation in Rocq. In \Cref{sec:jair-vm}, we define its semantics as a virtual machine.

\section{The \gls{jair} Syntax}
\label{sec:jair-syntax}

\gls{jair} leverages the type system of Rocq to rule out as many ill-formed \gls{ssa} programs as possible. We begin by describing how registers and labels are represented in our language. We then proceed in defining the different types of instructions, namely ALU-instructions, $\phi$-instructions, and jump instructions. Then, we define the basic block type.
At the end of the section we discuss the limitations of \gls{jair} by showing some examples.

\subsection{Registers}

Our implementation uses two kinds of registers: virtual registers in the earlier phases of \gls{ra}, and physical registers in the final phase. To support both virtual and physical registers, we make \gls{jair} parametric over the type of registers. To this end, we define the \texttt{IR} module. The \texttt{IR} module has the following signature.

\begin{lstlisting}[style=Rocq]
Module MakeIR (IR: IR_PARAMS).
  (* ... *)
\end{lstlisting}

Where \texttt{IR\_PARAMS} are the following parameters:
\begin{itemize}
    \item The register type \texttt{reg}, the set of all the registers we can choose from;
    \item A proof for the decidability of the logical equality for the register type, namely \texttt{reg\_eq\_dec};
\end{itemize}

Virtual registers use the set of natural numbers.

\begin{lstlisting}[style=Rocq]
Definition vreg := nat.
\end{lstlisting}

Physical registers, correspond to the 64-bit general-purpose registers of the x86-64 architecture.

\begin{lstlisting}[style=Rocq]
Inductive preg : Type :=
  | RAX
  | RBX
  | RCX
  (* ... *)
  | UNASSIGNED.
\end{lstlisting}

We also include the \texttt{UNASSIGNED} register, this register is used as a default value for the coloring, we explain the purpose of this register more in depth in \Cref{sec:ra}. Moreover, we will prove that the coloring function will never return this constructor, in \Cref{cha:verification}.

\subsection{Labels}

We now define the non-terminal of a block label:

\begin{align*}
\nt{label} ::= \text{Normal} \ \mathbb N \mid \text{Point1} \ \mathbb N \mid \text{Point2} \ \mathbb N
\end{align*}

Implemented in Rocq with the following definition:

\begin{lstlisting}[style=Rocq]
Inductive lbl : Type :=
  | Normal : nat -> lbl
  | Point1 : nat -> lbl
  | Point2 : nat -> lbl.
\end{lstlisting}

The label type has three constructors, \Cref{sec:destruct} discusses this more in depth. With the constructors \texttt{Point1} and \texttt{Point2} we identify additional basic blocks introduced in later stages of the pipeline, whereas with the \texttt{Normal} constructor we identify the blocks that already existed before \gls{ra}.

\subsection{ALU-Instructions}

In order to perform computations, \gls{jair} provides arithmetic and logical instructions.
We define the \texttt{val} type which represents an operand of these instructions. It can be either an immediate integer, a register, or a pointer to a memory location (represented by a natural number). The following is the \gls{bnf} definition of the value type, along with its type definition.

\begin{align*}
\nt{value} ::= i( \mathbb Z ) \mid r( \text{register} ) \mid p( \text{pointer} )
\end{align*}

\begin{lstlisting}[style=Rocq]
Inductive val : Type :=
  | Imm (x : Z)
  | Reg (r : reg)
  | Ptr (p : nat).
\end{lstlisting}

We define the syntax of expressions.

\begin{minipage}{0.45\textwidth}
\begin{align*}
\nt{expression} &::= \nt{value} \\
&\mid \text{load} \nt{value} \\
&\mid r( \nt{register}) + \nt{value} \\
&\mid r( \nt{register}) - \nt{value} \\
&\mid r( \nt{register}) * \nt{value} \\
&\mid r( \nt{register}) / \nt{value} \\
\end{align*}
\end{minipage}
\hfill
\begin{minipage}{0.45\textwidth}
\begin{lstlisting}[style=Rocq]
Inductive expr : Type :=
  | Val : val -> expr
  | Load : val -> expr
  | Add : reg -> val -> expr
  | Sub : reg -> val -> expr
  | Mul : reg -> val -> expr
  | Div : reg -> val -> expr.
\end{lstlisting}
\end{minipage}

These include registers, memory loads, and arithmetic and logical operations. The expression non-terminal is not defined recursively to preserve the linearity of the data structure.

As can be seen from the previous definitions, binary expressions always use a register as the first operand. This prevents expressions where all the operands are constants. In that case, in fact, the result could be computed during a previous step of constant folding. It should be also mentioned that, for the sake of \gls{ra}, a differentiation between unary, binary and $n$-ary expressions is unnecessary. This is because we are only concerned with the identities of the operands of an expression (the registers) and not the operators.

The instruction type \texttt{inst} reflects the core operations in our language. An instruction can either define a register by assigning to it the result of an expression, or it can store a value into memory.

\begin{align*}
\nt{instruction} &::= r( \nt{register} ) \leftarrow \nt{expression} \\
&\mid \text{store} \ r( \nt{register} ) \ r( \nt{register} )
\end{align*}

\begin{lstlisting}[style=Rocq]
Inductive inst : Type :=
  | Def (r : reg) (e : expr)
  | Store (v : val) (r : reg).
\end{lstlisting}

\texttt{Store} is treated specially, because it does not produce a result that can be assigned to a register. Therefore, it cannot be expressed as a \texttt{Def}.
Instructions are often grouped into sequences that are executed linearly. To represent it, we introduce the following non-terminals in our grammar.

\begin{align*}
\nt{instructions} &::= \nt{instruction} ; \ \nt{instructions} \mid \nt{instruction} \\
\nt{instructions-or-nil} &::= [ \nt{instructions} ] \mid []
\end{align*}

\subsection{$\phi$-Instructions}

\Cref{sec:ssa} defines a sequence of $\phi$-instructions as a parallel move based on the predecessor in the control flow.
We try to encode the data required by a $\phi$-instruction with the following definitions. We define the syntax of the arguments of a $\phi$-instruction.

\begin{align*}
\nt{phi-arguments} &::= ( \nt{register} , \ \nt{label} ) ; \ \nt{phi-arguments} \mid (\nt{register} , \ \nt{label}) \\
\nt{phi-arguments-or-nil} &::= [ \nt{phi-arguments} ] \mid []
\end{align*}

Here, even if $\phi$-instructions that have no arguments are considered ill-formed, we still allow them. This is because, in our internal representation, the arguments of a $\phi$-instruction are implemented as a list.
We define the correspondent type, for a single argument:

\begin{lstlisting}[style=Rocq]
Definition phi_arg : Type := (reg * lbl).
\end{lstlisting}

Semantically, the \texttt{reg} instance is the source of the copy iff the control flow comes from the \texttt{lbl} instance.

We define the syntax of the $\phi$-instruction, as an assignment of one of the possible $\nt{phi-arguments}$ to the destination $\nt{register}$.

\begin{align*}
\nt{phi} ::= r( \nt{register} ) \leftarrow \phi \ \nt{phi-arguments-or-nil}
\end{align*}

The same goes for the data type:

\begin{lstlisting}[style=Rocq]
Inductive phi : Type :=
| Phi : reg -> list phi_arg -> phi.
\end{lstlisting}

Much like ALU-instructions, $\phi$-instructions are often grouped into sequences. Because of that we introduce the following non-terminals.

\begin{align*}
\nt{phis} &::= \nt{phi} ; \ \nt{phis} \\
\nt{phis-or-nil} &::= [ \nt{phis} ] \mid []
\end{align*}

% Here, each \texttt{phi\_arg} pairs a register with a label identifying the originating block. Initially, we experimented with a label-free control flow representation that used direct pointers to blocks. However, as we will explain later, this made the semantics unnecessarily difficult to define, leading us to prefer label-based control flow.

\subsection{Blocks and Jump Instructions}

ALU-instructions and $\phi$-instructions are defined separately. This enforces the fact that, in well-formed basic blocks, the $\phi$-instructions appear first, followed by ALU-instructions, and, finally, by a jump instruction.
Before defining the syntax of a jump instruction, we introduce the conditional operators.

\begin{minipage}{0.45\linewidth}
\begin{align*}
\nt{condition} &::= r(\text{register}) = \nt{val} \\
&\mid r(\text{register}) \neq \nt{val} \\
&\mid r(\text{register}) < \nt{val} \\
&\mid r(\text{register}) \leq \nt{val} \\
&\mid r(\text{register}) > \nt{val} \\
&\mid r(\text{register}) \geq \nt{val} \\
\end{align*}
\end{minipage}
\hfill
\begin{minipage}{0.45\linewidth}
\begin{lstlisting}[style=Rocq]
Inductive cond : Type :=
  | CondEq
  | CondNe
  | CondLt
  | CondLe
  | CondGt
  | CondGe.
\end{lstlisting}
\end{minipage}

These operators represent the possible ways of comparing two items, we use them in the definition of the jump instruction.

\begin{align*}
\nt{jump-instruction} &::= \text{if} \ \nt{condition} \ \text{then} \ \nt{block} \ \text{else} \ \nt{block} \\
&\mid \text{jump} \ \nt{block} \\
&\mid \text{ret} \ r( \nt{register} )
\end{align*}

This definition uses the $\nt{block}$ non-terminal, representing a basic block. We define its syntax here:

\begin{align*}
\nt{block} ::= \text{Block} \ ( \nt{label} ) \ \nt{phis-or-nil} \ \nt{instructions-or-nil} \ ( \nt{jump-instruction} )
\end{align*}

Whereas, for the data structure, we implement the jump instruction type in mutual recursion with the block type.

\begin{lstlisting}[style=Rocq]
CoInductive block : Type :=
  | Block (l : lbl) (ps : list phi) (is : list inst) (j : jinst)
with jinst : Type :=
  | CondJump : cond -> reg -> val -> block -> block -> jinst
  | Jump : block -> jinst
  | Ret : reg -> jinst.
\end{lstlisting}

Control flow is encoded directly using block references instead of labels, preventing accidental jumps to nonexistent blocks. This definition naturally forms a \gls{cfg}, where the nodes are the basic blocks and jump instructions define the control flow relation. Finally, in order to define the starting point of the \gls{cfg}, we provide this definition.

\begin{lstlisting}[style=Rocq]
Definition program : Type := block.
\end{lstlisting}

Where the starting point is the first block we encounter during a visit of \texttt{program}, and the set of blocks of the \gls{cfg} is the set of blocks reachable from the starting point.

After covering the \gls{jair} implementation, we demonstrate its capabilities in \Cref{fig:example-jair} with an example of the Fibonacci function along with the same example in a \gls{ssa} form representation.

\begin{figure}[ht]
\centering
\begin{minipage}{0.65\textwidth}
\begin{lstlisting}[style=Rocq]
Definition b3 : block :=
  Block (Normal 3) [] [] (ret r(6)).

CoFixpoint b2 : block :=
  Block (Normal 2) [
    r(3) <- phi [(0, Normal 1); (4, Normal 2)];
    r(4) <- phi [(1, Normal 1); (6, Normal 2)];
    r(5) <- phi [(2, Normal 1); (7, Normal 2)]
  ] [
    r(6) <- r(4) + r(3);
    r(7) <- r(5) - i(1)
  ] (
    if r(7) = i(1)
    then b3
    else b2
  ).

Definition b1 : block :=
  Block (Normal 1) [] [
    r(0) <- i(0);  (* Second temp *)
    r(1) <- i(1);  (* First temp *)
    r(2) <- i(12)  (* Iterator *)
  ] (
    jump b2
  ).
\end{lstlisting}
\end{minipage}
\hfill
\begin{minipage}{0.30\textwidth}
\centering
\begin{tikzpicture}[
    node distance=10mm,
    every node/.style={draw, align=left, inner sep=4pt},
    >={Stealth}
]
  \node (entry)   at (0, 0)   {$r_0 \leftarrow 0$ \\ $r_1 \leftarrow 1$ \\ $r_2 \leftarrow 12$};
  \node (loop)    at (0, -3)  {$r_3 \leftarrow \phi(r_0, r_4)$ \\ $r_4 \leftarrow \phi(r_1, r_6)$ \\ $r_5 \leftarrow \phi(r_2, r_7)$ \\ $r_6 \leftarrow r_4 + r_3$ \\ $r_7 \leftarrow r_5 - 1$};
  \node (end)     at (0, -5.5)  {ret $r_6$};
  \draw[->] (entry) -- (loop);
  \draw[->] ([xshift=10pt]loop.south) to[out=315, in=45, looseness=4] ([xshift=10pt]loop.north);
  \draw[->] (loop) -- (end);
\end{tikzpicture}
\end{minipage}
\caption{Code for the Fibonacci function in \gls{jair} along with the same code in a \gls{ssa} form representation}
\label{fig:example-jair}
\end{figure}

\subsection{Limitations}
\label{subsec:limitations}

Initially, we opted for a label-less representation of basic blocks.
Later in the project we saw the necessity of introducing them. Take the following example of a label-less program:

\begin{lstlisting}[style=Rocq]
Definition b3 : block :=
  Block [r(3) <- phi [1, 2]] [] (ret r(3)).
Definition b2 : block :=
  Block [] [r(2) <- i(2)] (Jump b3).
Definition b1 : block :=
  Block [] [r(1) <- i(1)] (Jump b2).
\end{lstlisting}

At line 2, we use a $\phi$-instruction. However, we do not encode the order of the predecessors \texttt{b1} and \texttt{b2}. Because of that, the $\phi$-instruction cannot know which of the two arguments belongs in which predecessor. In this simple case, it is immediate to see that \texttt{r1} belongs to \texttt{b1}, and \texttt{r2} belongs to \texttt{b2}. Even so, we must keep in mind that some $\phi$-instructions may use variables that are not defined in their direct predecessors. For this reason, we adopted labels at the cost of introducing possible inconsistencies in the representation. We will now talk about these inconsistencies.

First, uniqueness of labels is not enforced. This can lead to conflicting block labels, as shown in the following example, where we define two blocks with label \texttt{(Normal 0)}.

\begin{lstlisting}[style=Rocq]
CoFixpoint b1 : block :=
  Block (Normal 0) [] [] (jump b2)
with b2 : block :=
  Block (Normal 0) [] [] (jump b1).
\end{lstlisting}

Another problem is that $\phi$-instructions may include inconsistent or invalid arguments. In the following example, the number of predecessors, listed in the $\phi$-instruction at line 2, does not match the actual number of predecessors. Furthermore, the labels referenced by the $\phi$-instruction do not exist:

\begin{lstlisting}[style=Rocq]
Definition b2 : block :=
  Block (Normal 1) [
    r(1) <- phi [(1, 5); (2, 6); (3; 7)]] [] (ret r(0)).
Definition b1 : block :=
  Block (Normal 0) [] [] (jump b2).
\end{lstlisting}

Despite our efforts to rule out incorrect \gls{ssa} programs, certain ill-formed programs cannot be excluded purely by the syntax. We now provide examples of such cases.

A fundamental problem is that the two core policies of a \gls{ssa} form representation (see \Cref{def:ssa}) are not enforced. In particular, multiple assignments to the same register are allowed. Take the following example:

\begin{lstlisting}[style=Rocq]
Definition b1 : block :=
  Block (Normal 0) [] [
    r(1) <- i(0);
    r(1) <- i(1)
  ]
  (ret r(1)).
\end{lstlisting}

At line 4, we define variable \texttt{r(1)} for a second time.
Moreover, the usage of undefined variables is admitted:

\begin{lstlisting}[style=Rocq]
Definition b1 : block :=
  Block (Normal 0) [] [r(1) <- r(0)] (ret r(1)).
\end{lstlisting}

In the previous snippet, at line 2, we define \texttt{r(1)}, using variable \texttt{r(0)}, which is undefined.
Although these issues cannot be resolved by the type system alone, they can be caught during a previous phase of semantic analysis.

\section{A \gls{jair} Interpreter}
\label{sec:jair-vm}

We now implement an interpreter for our intermediate representation. We do this for two reasons. The first one is to clearly state the semantics of \gls{jair}. The second one has to do with the testing of our \gls{ra} pipeline. In \Cref{cha:extraction}, we compare the result of a program executed on the interpreter virtual machine with the result of the same program executed on bare hardware. If the two results match this suggests that the semantics of the program are preserved even after \gls{ra}.
The definition of the virtual machine is straightforward as its components are just the register file, implemented as a map from registers to integers, and the memory, implemented as list of integers.

\begin{lstlisting}[style=Rocq]
Inductive vm : Type :=
  | Vm : (reg -> Z) -> list Z -> vm.
\end{lstlisting}

After defining the virtual machine, we define the primitives to interact with it. We begin with operations on the register file.

The function to read a value from a register is defined as follows:

\begin{lstlisting}[style=Rocq]
Definition get_reg (m : vm) (r : reg) : cell :=
  match m with
  | Vm regs _ => regs r
  end.
\end{lstlisting}

This function takes a virtual machine state \texttt m and a register \texttt r, and returns the content of the register  by performing $\beta$-reduction on the register file map.

The corresponding function to write a value into a register is defined as:

\begin{lstlisting}[style=Rocq]
Definition set_reg (m : vm) (r : reg) (c : cell) : vm :=
  match m with
  | Vm regs cells =>
    Vm (fun r' => if r' =? r then c else regs r') cells
  end.
\end{lstlisting}

Here, we take the current register file map and we wrap it with a lambda function, if the argument of that function matches register \texttt r, we return the value \texttt c, otherwise we perform $\beta$-reduction with the previous version of the register file map.

Now, we define analogous operations for memory access. In our implementation memory does not support random access. Instead, we use a list that grows whenever a new value is stored, filling the previously uninitialized memory locations with zeroes. The function to retrieve the content of a memory location is defined as follows:

\begin{lstlisting}[style=Rocq]
Definition get_cell (m : vm) (i : nat) : cell :=
  let fix get_cell_aux (cells : list cell) (i : nat) : cell :=
    match cells, i with
    | nil, _ => Z0
    | c :: _, O => c
    | _ :: cs, S i' => get_cell_aux cs i'
    end
  in
  match m with
  | Vm _ cells => get_cell_aux cells i
  end.
\end{lstlisting}

The recursive function considers three cases:
\begin{itemize}
  \item If the memory is empty, then the value we are trying to retrieve is Zero;
  \item If the index is Zero, we return the value of the current memory location;
  \item Otherwise, if we have not reached the desired address yet, we perform a recursive call on the remaining memory;
\end{itemize}

We define the function for writing a value to a memory location:

\begin{lstlisting}[style=Rocq]
Definition set_cell (m : vm) (i : nat) (c : cell) : vm :=
  let fix set_cell_aux (cells : list cell) (i : nat) (c : cell) : list cell :=
    match cells, i with
    | nil, O => c :: nil
    | nil, S i' => Z0 :: (set_cell_aux nil i' c)
    | _ :: xs, O => c :: xs
    | x :: xs, S i' => x :: (set_cell_aux xs i' c)
    end
  in
  match m with
  | Vm regs cells => Vm regs (set_cell_aux cells i c)
  end.
\end{lstlisting}

The recursive function handles four cases:
\begin{itemize}
  \item If the memory is empty, but we reached the intended address, we expand the memory with one last cell where we store the content;
  \item If the memory is empty, and we have not reached the desired address yet, we create an empty cell and make a recursive call;
  \item If we reached the intended address, we overwrite the current content with the new content;
  \item Finally, if we have not reached the intended address yet, we perform a recursive call on the remaining memory;
\end{itemize}

The remaining task is to define a function that simulates program execution, ideally we would want that function to have type \texttt{vm $\to$ program $\to$ vm} taking the initial state of the virtual machine and the program as inputs and returning the new state of the virtual machine.
Unfortunately, as we mentioned in \Cref{subsec:funterm}, Rocq does not allow for the definition of functions of which we cannot prove termination, and proving it for this function would entail solving the halting problem. We thus resort to using fuel based recursion for this case.

\begin{lstlisting}[style=Rocq]
Fixpoint run (m : vm) (p : program) (fuel : nat) : vm :=
  match p, fuel with
  | _, O => m
  | Block _ _ is j, S fuel' =>
    let m := run_insts m is in
    match j with
    | CondJump c r v b1 b2 =>
      if eval_cond m c r v then
        run (run_phis m p b1) b1 fuel'
      else
        run (run_phis m p b2) b2 fuel'
    | Jump b1 => run (run_phis m p b1) b1 fuel'
    | Ret r => set_reg m 0 (get_reg m r)
    end
  end.
\end{lstlisting}

The procedure works this way: if we are out of fuel, we terminate, and we return the current state of the virtual machine. Otherwise, we proceed with the computation.
We run the body of instructions of the current block with the function \texttt{run\_insts}, yielding a new state of the virtual machine. We then treat each jump instruction differently. In the case of a conditional jump we evaluate the condition with the \texttt{eval\_cond} function. Instead, in the other two cases, no evaluation is needed. Finally, before jumping to the next block we execute its $\phi$-instructions. We do that by call the \texttt{run\_phis} function passing as arguments the virtual machine, the current block, and the successor block.
We continue with the computation until we either run out of fuel, or we reach the \texttt{Ret} instruction, saving the result of the program into the first register.

% \chapter[Classificatore Bayesiano e $K$-Nearest Neighbor]{Classificatore Bayesiano e $K$-Nearest Neighbor} \label{chapter bayes}
% \chaptermark{Classificatore Bayesiano e $K$NN}
% % Errors

\chapter{Background}
\label{cha:background}

In this chapter, we present the foundational tools and concepts employed throughout this project. We begin by introducing the Coq Proof Assistant, highlighting the features that make it particularly suitable for our implementation. We then describe the Single Static Assignment (SSA) form, the intermediate representation used in our compiler, along with the necessary structural elements to make it practical. Finally, we discuss the theoretical foundations of register allocation, with particular attention to its formulation within the SSA paradigm.

\section{The Coq Proof Assistant}

The core of our register allocator is implemented in \textit{Coq}, an interactive proof assistant based on a dependently typed functional language known as \textit{Gallina}. While Coq is primarily used for formal verification, in this project we focus more on its computational capabilities than on the formal proofs of correctness.

\subsection{Function Termination}
\label{subsec:funterm}

A fundamental feature of Coq is its enforcement of termination in all function definitions. Recursive functions are typically defined using the \texttt{Fixpoint} keyword, which requires pattern matching over inductive types. Coq statically checks that, for each recursive call, there exists a structurally decreasing argument, guaranteeing termination by eventually reaching a base case.

Some recursive functions, however, do not conform to this syntactic criterion. In such cases, Coq offers two common workarounds:

\begin{enumerate}
    \item \textbf{Fuel-based recursion}: This approach involves adding an artificial decreasing argument (commonly named \texttt{fuel} of type \texttt{nat}) to the function signature. The function then proceeds recursively while decrementing \texttt{fuel} at each step. Termination is guaranteed when \texttt{fuel} reaches zero. The drawback is that an upper bound on the number of iterations must be known in advance, which may be difficult to determine and, if underestimated, can lead to incomplete computations.
    
    \item \textbf{Using \texttt{Function} with a termination proof}: This alternative allows defining more general recursive functions while providing an explicit proof of their termination. Although more flexible, this method introduces additional proof obligations and complexity.
\end{enumerate}

In this project, we primarily adopt the fuel-based approach, assuming that appropriate bounds on the number of iterations are known for the algorithms in question.

\subsection{OCaml Extraction}
% Content to be added.

\section{Single Static Assignment (SSA)}
\label{sec:ssa}

We adopt the Single Static Assignment (SSA) form as our intermediate representation, due to its structural properties that greatly simplify register allocation. An SSA-form program satisfies the following constraints:

\begin{itemize}
    \item Each variable (or virtual register) is assigned exactly once;
    \item Every use of a variable is dominated by its unique definition.
\end{itemize}

A label $l_1$ is said to \textit{dominate} a label $l_2$ if every path from the entry point of the control flow graph (CFG) to $l_2$ passes through $l_1$.

To make SSA a practical intermediate representation, we must introduce additional structures, namely the control flow graph and $\phi$-functions.

\subsection{Control Flow Graphs}
\label{subsec:cfg}

To represent the control flow of programs, we employ Control Flow Graphs (CFGs). A CFG is defined as the triple $(B, CF, \textbf{start})$, where:

\begin{itemize}
    \item $B$ is the set of \textit{basic blocks}, each consisting of a sequence of instructions preceded by a $\phi$ section and terminated by a control transfer instruction (e.g., branch or jump).
    \item $CF \subseteq B \times B$ is the control flow relation, indicating allowed transitions between blocks.
    \item $\textbf{start} \in B$ is the designated entry point of the program.
\end{itemize}

\subsection{Phi Operations}
\label{subsec:phi}

% Due to the single-assignment constraint in SSA, merging variable values from multiple control paths requires the use of \textit{$\phi$-functions}. A $\phi$-instruction has the general form:

% \[
% x \leftarrow \phi((y_1, b_1), (y_2, b_2), \dots, (y_n, b_n))
% \]

% Here, each pair $(y_i, b_i)$ indicates that the value of $y_i$ should be assigned to $x$ if the program arrives from block $b_i$. $\phi$-functions are always placed at the beginning of a basic block, and are semantically executed during the control transfer from a predecessor block.

% We often group multiple $\phi$-assignments into a matrix-like section:

% \[
% \begin{pmatrix}
%     x_1 \\ x_2 \\ \vdots \\ x_m
% \end{pmatrix}
% \leftarrow \phi
% \begin{pmatrix}
%     (y_{11}, b_1) & (y_{12}, b_2) & \dots & (y_{1n}, b_n) \\
%     (y_{21}, b_1) & (y_{22}, b_2) & \dots & (y_{2n}, b_n) \\
%     \vdots & \vdots & \ddots & \vdots \\
%     (y_{m1}, b_1) & (y_{m2}, b_2) & \dots & (y_{mn}, b_n) \\
% \end{pmatrix}
% \]

% An important semantic property is that all $\phi$-assignments in a block are executed \textit{in parallel}. This aspect must be carefully preserved when translating SSA code to machine code.

\section{Register Allocation}
\label{sec:ra}

Register allocation is the process of assigning a potentially unbounded number of program variables to a limited number of physical registers. A well-known approach reduces this problem to graph coloring.

Given a program, we construct its \textit{interference graph} $G = (V, E)$, where:

\begin{itemize}
    \item $V$ is the set of variables (or virtual registers);
    \item $(u, v) \in E$ if variables $u$ and $v$ are simultaneously live at some program point, and therefore cannot share the same register.
\end{itemize}

A valid $k$-coloring of the interference graph, where each color represents a physical register, corresponds to a feasible register assignment for a machine with $k$ registers.

\subsection{Register Allocation in SSA Form}

While graph coloring is $\mathcal{NP}$-complete for arbitrary graphs, the interference graphs of programs in SSA form have a special structure: they are \textit{chordal} graphs. A chordal graph is one in which every cycle of four or more vertices has a chord, i.e., an edge connecting two non-consecutive vertices.

This structural property allows optimal register allocation to be performed in polynomial time, specifically, in quadratic time, making SSA form not only more analyzable but also more efficient for code generation tasks such as register allocation.



% \chapter{Regressione Logistica} \label{chapter logreg}
% % Errors

\chapter{\gls{jair}}
\label{cha:jair}

Here, we detail the implementation of \gls{jair}. In particular, in \Cref{sec:jair-syntax}, we define the syntax of the language in \gls{bnf}, and we show its implementation in Rocq. In \Cref{sec:jair-vm}, we define its semantics as a virtual machine.

\section{The \gls{jair} Syntax}
\label{sec:jair-syntax}

\gls{jair} leverages the type system of Rocq to rule out as many ill-formed \gls{ssa} programs as possible. We begin by describing how registers and labels are represented in our language. We then proceed in defining the different types of instructions, namely ALU-instructions, $\phi$-instructions, and jump instructions. Then, we define the basic block type.
At the end of the section we discuss the limitations of \gls{jair} by showing some examples.

\subsection{Registers}

Our implementation uses two kinds of registers: virtual registers in the earlier phases of \gls{ra}, and physical registers in the final phase. To support both virtual and physical registers, we make \gls{jair} parametric over the type of registers. To this end, we define the \texttt{IR} module. The \texttt{IR} module has the following signature.

\begin{lstlisting}[style=Rocq]
Module MakeIR (IR: IR_PARAMS).
  (* ... *)
\end{lstlisting}

Where \texttt{IR\_PARAMS} are the following parameters:
\begin{itemize}
    \item The register type \texttt{reg}, the set of all the registers we can choose from;
    \item A proof for the decidability of the logical equality for the register type, namely \texttt{reg\_eq\_dec};
\end{itemize}

Virtual registers use the set of natural numbers.

\begin{lstlisting}[style=Rocq]
Definition vreg := nat.
\end{lstlisting}

Physical registers, correspond to the 64-bit general-purpose registers of the x86-64 architecture.

\begin{lstlisting}[style=Rocq]
Inductive preg : Type :=
  | RAX
  | RBX
  | RCX
  (* ... *)
  | UNASSIGNED.
\end{lstlisting}

We also include the \texttt{UNASSIGNED} register, this register is used as a default value for the coloring, we explain the purpose of this register more in depth in \Cref{sec:ra}. Moreover, we will prove that the coloring function will never return this constructor, in \Cref{cha:verification}.

\subsection{Labels}

We now define the non-terminal of a block label:

\begin{align*}
\nt{label} ::= \text{Normal} \ \mathbb N \mid \text{Point1} \ \mathbb N \mid \text{Point2} \ \mathbb N
\end{align*}

Implemented in Rocq with the following definition:

\begin{lstlisting}[style=Rocq]
Inductive lbl : Type :=
  | Normal : nat -> lbl
  | Point1 : nat -> lbl
  | Point2 : nat -> lbl.
\end{lstlisting}

The label type has three constructors, \Cref{sec:destruct} discusses this more in depth. With the constructors \texttt{Point1} and \texttt{Point2} we identify additional basic blocks introduced in later stages of the pipeline, whereas with the \texttt{Normal} constructor we identify the blocks that already existed before \gls{ra}.

\subsection{ALU-Instructions}

In order to perform computations, \gls{jair} provides arithmetic and logical instructions.
We define the \texttt{val} type which represents an operand of these instructions. It can be either an immediate integer, a register, or a pointer to a memory location (represented by a natural number). The following is the \gls{bnf} definition of the value type, along with its type definition.

\begin{align*}
\nt{value} ::= i( \mathbb Z ) \mid r( \text{register} ) \mid p( \text{pointer} )
\end{align*}

\begin{lstlisting}[style=Rocq]
Inductive val : Type :=
  | Imm (x : Z)
  | Reg (r : reg)
  | Ptr (p : nat).
\end{lstlisting}

We define the syntax of expressions.

\begin{minipage}{0.45\textwidth}
\begin{align*}
\nt{expression} &::= \nt{value} \\
&\mid \text{load} \nt{value} \\
&\mid r( \nt{register}) + \nt{value} \\
&\mid r( \nt{register}) - \nt{value} \\
&\mid r( \nt{register}) * \nt{value} \\
&\mid r( \nt{register}) / \nt{value} \\
\end{align*}
\end{minipage}
\hfill
\begin{minipage}{0.45\textwidth}
\begin{lstlisting}[style=Rocq]
Inductive expr : Type :=
  | Val : val -> expr
  | Load : val -> expr
  | Add : reg -> val -> expr
  | Sub : reg -> val -> expr
  | Mul : reg -> val -> expr
  | Div : reg -> val -> expr.
\end{lstlisting}
\end{minipage}

These include registers, memory loads, and arithmetic and logical operations. The expression non-terminal is not defined recursively to preserve the linearity of the data structure.

As can be seen from the previous definitions, binary expressions always use a register as the first operand. This prevents expressions where all the operands are constants. In that case, in fact, the result could be computed during a previous step of constant folding. It should be also mentioned that, for the sake of \gls{ra}, a differentiation between unary, binary and $n$-ary expressions is unnecessary. This is because we are only concerned with the identities of the operands of an expression (the registers) and not the operators.

The instruction type \texttt{inst} reflects the core operations in our language. An instruction can either define a register by assigning to it the result of an expression, or it can store a value into memory.

\begin{align*}
\nt{instruction} &::= r( \nt{register} ) \leftarrow \nt{expression} \\
&\mid \text{store} \ r( \nt{register} ) \ r( \nt{register} )
\end{align*}

\begin{lstlisting}[style=Rocq]
Inductive inst : Type :=
  | Def (r : reg) (e : expr)
  | Store (v : val) (r : reg).
\end{lstlisting}

\texttt{Store} is treated specially, because it does not produce a result that can be assigned to a register. Therefore, it cannot be expressed as a \texttt{Def}.
Instructions are often grouped into sequences that are executed linearly. To represent it, we introduce the following non-terminals in our grammar.

\begin{align*}
\nt{instructions} &::= \nt{instruction} ; \ \nt{instructions} \mid \nt{instruction} \\
\nt{instructions-or-nil} &::= [ \nt{instructions} ] \mid []
\end{align*}

\subsection{$\phi$-Instructions}

\Cref{sec:ssa} defines a sequence of $\phi$-instructions as a parallel move based on the predecessor in the control flow.
We try to encode the data required by a $\phi$-instruction with the following definitions. We define the syntax of the arguments of a $\phi$-instruction.

\begin{align*}
\nt{phi-arguments} &::= ( \nt{register} , \ \nt{label} ) ; \ \nt{phi-arguments} \mid (\nt{register} , \ \nt{label}) \\
\nt{phi-arguments-or-nil} &::= [ \nt{phi-arguments} ] \mid []
\end{align*}

Here, even if $\phi$-instructions that have no arguments are considered ill-formed, we still allow them. This is because, in our internal representation, the arguments of a $\phi$-instruction are implemented as a list.
We define the correspondent type, for a single argument:

\begin{lstlisting}[style=Rocq]
Definition phi_arg : Type := (reg * lbl).
\end{lstlisting}

Semantically, the \texttt{reg} instance is the source of the copy iff the control flow comes from the \texttt{lbl} instance.

We define the syntax of the $\phi$-instruction, as an assignment of one of the possible $\nt{phi-arguments}$ to the destination $\nt{register}$.

\begin{align*}
\nt{phi} ::= r( \nt{register} ) \leftarrow \phi \ \nt{phi-arguments-or-nil}
\end{align*}

The same goes for the data type:

\begin{lstlisting}[style=Rocq]
Inductive phi : Type :=
| Phi : reg -> list phi_arg -> phi.
\end{lstlisting}

Much like ALU-instructions, $\phi$-instructions are often grouped into sequences. Because of that we introduce the following non-terminals.

\begin{align*}
\nt{phis} &::= \nt{phi} ; \ \nt{phis} \\
\nt{phis-or-nil} &::= [ \nt{phis} ] \mid []
\end{align*}

% Here, each \texttt{phi\_arg} pairs a register with a label identifying the originating block. Initially, we experimented with a label-free control flow representation that used direct pointers to blocks. However, as we will explain later, this made the semantics unnecessarily difficult to define, leading us to prefer label-based control flow.

\subsection{Blocks and Jump Instructions}

ALU-instructions and $\phi$-instructions are defined separately. This enforces the fact that, in well-formed basic blocks, the $\phi$-instructions appear first, followed by ALU-instructions, and, finally, by a jump instruction.
Before defining the syntax of a jump instruction, we introduce the conditional operators.

\begin{minipage}{0.45\linewidth}
\begin{align*}
\nt{condition} &::= r(\text{register}) = \nt{val} \\
&\mid r(\text{register}) \neq \nt{val} \\
&\mid r(\text{register}) < \nt{val} \\
&\mid r(\text{register}) \leq \nt{val} \\
&\mid r(\text{register}) > \nt{val} \\
&\mid r(\text{register}) \geq \nt{val} \\
\end{align*}
\end{minipage}
\hfill
\begin{minipage}{0.45\linewidth}
\begin{lstlisting}[style=Rocq]
Inductive cond : Type :=
  | CondEq
  | CondNe
  | CondLt
  | CondLe
  | CondGt
  | CondGe.
\end{lstlisting}
\end{minipage}

These operators represent the possible ways of comparing two items, we use them in the definition of the jump instruction.

\begin{align*}
\nt{jump-instruction} &::= \text{if} \ \nt{condition} \ \text{then} \ \nt{block} \ \text{else} \ \nt{block} \\
&\mid \text{jump} \ \nt{block} \\
&\mid \text{ret} \ r( \nt{register} )
\end{align*}

This definition uses the $\nt{block}$ non-terminal, representing a basic block. We define its syntax here:

\begin{align*}
\nt{block} ::= \text{Block} \ ( \nt{label} ) \ \nt{phis-or-nil} \ \nt{instructions-or-nil} \ ( \nt{jump-instruction} )
\end{align*}

Whereas, for the data structure, we implement the jump instruction type in mutual recursion with the block type.

\begin{lstlisting}[style=Rocq]
CoInductive block : Type :=
  | Block (l : lbl) (ps : list phi) (is : list inst) (j : jinst)
with jinst : Type :=
  | CondJump : cond -> reg -> val -> block -> block -> jinst
  | Jump : block -> jinst
  | Ret : reg -> jinst.
\end{lstlisting}

Control flow is encoded directly using block references instead of labels, preventing accidental jumps to nonexistent blocks. This definition naturally forms a \gls{cfg}, where the nodes are the basic blocks and jump instructions define the control flow relation. Finally, in order to define the starting point of the \gls{cfg}, we provide this definition.

\begin{lstlisting}[style=Rocq]
Definition program : Type := block.
\end{lstlisting}

Where the starting point is the first block we encounter during a visit of \texttt{program}, and the set of blocks of the \gls{cfg} is the set of blocks reachable from the starting point.

After covering the \gls{jair} implementation, we demonstrate its capabilities in \Cref{fig:example-jair} with an example of the Fibonacci function along with the same example in a \gls{ssa} form representation.

\begin{figure}[ht]
\centering
\begin{minipage}{0.65\textwidth}
\begin{lstlisting}[style=Rocq]
Definition b3 : block :=
  Block (Normal 3) [] [] (ret r(6)).

CoFixpoint b2 : block :=
  Block (Normal 2) [
    r(3) <- phi [(0, Normal 1); (4, Normal 2)];
    r(4) <- phi [(1, Normal 1); (6, Normal 2)];
    r(5) <- phi [(2, Normal 1); (7, Normal 2)]
  ] [
    r(6) <- r(4) + r(3);
    r(7) <- r(5) - i(1)
  ] (
    if r(7) = i(1)
    then b3
    else b2
  ).

Definition b1 : block :=
  Block (Normal 1) [] [
    r(0) <- i(0);  (* Second temp *)
    r(1) <- i(1);  (* First temp *)
    r(2) <- i(12)  (* Iterator *)
  ] (
    jump b2
  ).
\end{lstlisting}
\end{minipage}
\hfill
\begin{minipage}{0.30\textwidth}
\centering
\begin{tikzpicture}[
    node distance=10mm,
    every node/.style={draw, align=left, inner sep=4pt},
    >={Stealth}
]
  \node (entry)   at (0, 0)   {$r_0 \leftarrow 0$ \\ $r_1 \leftarrow 1$ \\ $r_2 \leftarrow 12$};
  \node (loop)    at (0, -3)  {$r_3 \leftarrow \phi(r_0, r_4)$ \\ $r_4 \leftarrow \phi(r_1, r_6)$ \\ $r_5 \leftarrow \phi(r_2, r_7)$ \\ $r_6 \leftarrow r_4 + r_3$ \\ $r_7 \leftarrow r_5 - 1$};
  \node (end)     at (0, -5.5)  {ret $r_6$};
  \draw[->] (entry) -- (loop);
  \draw[->] ([xshift=10pt]loop.south) to[out=315, in=45, looseness=4] ([xshift=10pt]loop.north);
  \draw[->] (loop) -- (end);
\end{tikzpicture}
\end{minipage}
\caption{Code for the Fibonacci function in \gls{jair} along with the same code in a \gls{ssa} form representation}
\label{fig:example-jair}
\end{figure}

\subsection{Limitations}
\label{subsec:limitations}

Initially, we opted for a label-less representation of basic blocks.
Later in the project we saw the necessity of introducing them. Take the following example of a label-less program:

\begin{lstlisting}[style=Rocq]
Definition b3 : block :=
  Block [r(3) <- phi [1, 2]] [] (ret r(3)).
Definition b2 : block :=
  Block [] [r(2) <- i(2)] (Jump b3).
Definition b1 : block :=
  Block [] [r(1) <- i(1)] (Jump b2).
\end{lstlisting}

At line 2, we use a $\phi$-instruction. However, we do not encode the order of the predecessors \texttt{b1} and \texttt{b2}. Because of that, the $\phi$-instruction cannot know which of the two arguments belongs in which predecessor. In this simple case, it is immediate to see that \texttt{r1} belongs to \texttt{b1}, and \texttt{r2} belongs to \texttt{b2}. Even so, we must keep in mind that some $\phi$-instructions may use variables that are not defined in their direct predecessors. For this reason, we adopted labels at the cost of introducing possible inconsistencies in the representation. We will now talk about these inconsistencies.

First, uniqueness of labels is not enforced. This can lead to conflicting block labels, as shown in the following example, where we define two blocks with label \texttt{(Normal 0)}.

\begin{lstlisting}[style=Rocq]
CoFixpoint b1 : block :=
  Block (Normal 0) [] [] (jump b2)
with b2 : block :=
  Block (Normal 0) [] [] (jump b1).
\end{lstlisting}

Another problem is that $\phi$-instructions may include inconsistent or invalid arguments. In the following example, the number of predecessors, listed in the $\phi$-instruction at line 2, does not match the actual number of predecessors. Furthermore, the labels referenced by the $\phi$-instruction do not exist:

\begin{lstlisting}[style=Rocq]
Definition b2 : block :=
  Block (Normal 1) [
    r(1) <- phi [(1, 5); (2, 6); (3; 7)]] [] (ret r(0)).
Definition b1 : block :=
  Block (Normal 0) [] [] (jump b2).
\end{lstlisting}

Despite our efforts to rule out incorrect \gls{ssa} programs, certain ill-formed programs cannot be excluded purely by the syntax. We now provide examples of such cases.

A fundamental problem is that the two core policies of a \gls{ssa} form representation (see \Cref{def:ssa}) are not enforced. In particular, multiple assignments to the same register are allowed. Take the following example:

\begin{lstlisting}[style=Rocq]
Definition b1 : block :=
  Block (Normal 0) [] [
    r(1) <- i(0);
    r(1) <- i(1)
  ]
  (ret r(1)).
\end{lstlisting}

At line 4, we define variable \texttt{r(1)} for a second time.
Moreover, the usage of undefined variables is admitted:

\begin{lstlisting}[style=Rocq]
Definition b1 : block :=
  Block (Normal 0) [] [r(1) <- r(0)] (ret r(1)).
\end{lstlisting}

In the previous snippet, at line 2, we define \texttt{r(1)}, using variable \texttt{r(0)}, which is undefined.
Although these issues cannot be resolved by the type system alone, they can be caught during a previous phase of semantic analysis.

\section{A \gls{jair} Interpreter}
\label{sec:jair-vm}

We now implement an interpreter for our intermediate representation. We do this for two reasons. The first one is to clearly state the semantics of \gls{jair}. The second one has to do with the testing of our \gls{ra} pipeline. In \Cref{cha:extraction}, we compare the result of a program executed on the interpreter virtual machine with the result of the same program executed on bare hardware. If the two results match this suggests that the semantics of the program are preserved even after \gls{ra}.
The definition of the virtual machine is straightforward as its components are just the register file, implemented as a map from registers to integers, and the memory, implemented as list of integers.

\begin{lstlisting}[style=Rocq]
Inductive vm : Type :=
  | Vm : (reg -> Z) -> list Z -> vm.
\end{lstlisting}

After defining the virtual machine, we define the primitives to interact with it. We begin with operations on the register file.

The function to read a value from a register is defined as follows:

\begin{lstlisting}[style=Rocq]
Definition get_reg (m : vm) (r : reg) : cell :=
  match m with
  | Vm regs _ => regs r
  end.
\end{lstlisting}

This function takes a virtual machine state \texttt m and a register \texttt r, and returns the content of the register  by performing $\beta$-reduction on the register file map.

The corresponding function to write a value into a register is defined as:

\begin{lstlisting}[style=Rocq]
Definition set_reg (m : vm) (r : reg) (c : cell) : vm :=
  match m with
  | Vm regs cells =>
    Vm (fun r' => if r' =? r then c else regs r') cells
  end.
\end{lstlisting}

Here, we take the current register file map and we wrap it with a lambda function, if the argument of that function matches register \texttt r, we return the value \texttt c, otherwise we perform $\beta$-reduction with the previous version of the register file map.

Now, we define analogous operations for memory access. In our implementation memory does not support random access. Instead, we use a list that grows whenever a new value is stored, filling the previously uninitialized memory locations with zeroes. The function to retrieve the content of a memory location is defined as follows:

\begin{lstlisting}[style=Rocq]
Definition get_cell (m : vm) (i : nat) : cell :=
  let fix get_cell_aux (cells : list cell) (i : nat) : cell :=
    match cells, i with
    | nil, _ => Z0
    | c :: _, O => c
    | _ :: cs, S i' => get_cell_aux cs i'
    end
  in
  match m with
  | Vm _ cells => get_cell_aux cells i
  end.
\end{lstlisting}

The recursive function considers three cases:
\begin{itemize}
  \item If the memory is empty, then the value we are trying to retrieve is Zero;
  \item If the index is Zero, we return the value of the current memory location;
  \item Otherwise, if we have not reached the desired address yet, we perform a recursive call on the remaining memory;
\end{itemize}

We define the function for writing a value to a memory location:

\begin{lstlisting}[style=Rocq]
Definition set_cell (m : vm) (i : nat) (c : cell) : vm :=
  let fix set_cell_aux (cells : list cell) (i : nat) (c : cell) : list cell :=
    match cells, i with
    | nil, O => c :: nil
    | nil, S i' => Z0 :: (set_cell_aux nil i' c)
    | _ :: xs, O => c :: xs
    | x :: xs, S i' => x :: (set_cell_aux xs i' c)
    end
  in
  match m with
  | Vm regs cells => Vm regs (set_cell_aux cells i c)
  end.
\end{lstlisting}

The recursive function handles four cases:
\begin{itemize}
  \item If the memory is empty, but we reached the intended address, we expand the memory with one last cell where we store the content;
  \item If the memory is empty, and we have not reached the desired address yet, we create an empty cell and make a recursive call;
  \item If we reached the intended address, we overwrite the current content with the new content;
  \item Finally, if we have not reached the intended address yet, we perform a recursive call on the remaining memory;
\end{itemize}

The remaining task is to define a function that simulates program execution, ideally we would want that function to have type \texttt{vm $\to$ program $\to$ vm} taking the initial state of the virtual machine and the program as inputs and returning the new state of the virtual machine.
Unfortunately, as we mentioned in \Cref{subsec:funterm}, Rocq does not allow for the definition of functions of which we cannot prove termination, and proving it for this function would entail solving the halting problem. We thus resort to using fuel based recursion for this case.

\begin{lstlisting}[style=Rocq]
Fixpoint run (m : vm) (p : program) (fuel : nat) : vm :=
  match p, fuel with
  | _, O => m
  | Block _ _ is j, S fuel' =>
    let m := run_insts m is in
    match j with
    | CondJump c r v b1 b2 =>
      if eval_cond m c r v then
        run (run_phis m p b1) b1 fuel'
      else
        run (run_phis m p b2) b2 fuel'
    | Jump b1 => run (run_phis m p b1) b1 fuel'
    | Ret r => set_reg m 0 (get_reg m r)
    end
  end.
\end{lstlisting}

The procedure works this way: if we are out of fuel, we terminate, and we return the current state of the virtual machine. Otherwise, we proceed with the computation.
We run the body of instructions of the current block with the function \texttt{run\_insts}, yielding a new state of the virtual machine. We then treat each jump instruction differently. In the case of a conditional jump we evaluate the condition with the \texttt{eval\_cond} function. Instead, in the other two cases, no evaluation is needed. Finally, before jumping to the next block we execute its $\phi$-instructions. We do that by call the \texttt{run\_phis} function passing as arguments the virtual machine, the current block, and the successor block.
We continue with the computation until we either run out of fuel, or we reach the \texttt{Ret} instruction, saving the result of the program into the first register.

% \chapter[Modelli generativi per la classificazione]{Modelli generativi per la classificazione} \label{chapter modelli generativi}
% \chaptermark{Modelli generativi}
% % Errors

\chapter{\glsentrylong{ra}}
\label{cha:ra}

\section{Liveness Analysis}
\label{sec:liveness}

In this phase, we compute the set of live variables at each program point. This step is crucial for determining which variables interfere and, therefore constitute an edge of the interference graph.

Generally a liveness analysis algorithm works this way:
we start from the last instruction of the program and go back to the start instruction in a post-order fashion, while we do that we compute the following sets for each instruction $i$:
\begin{itemize}
  \item LiveIn[$i$] which contains the variables that are live \textit{before} the execution of instruction $i$;
  \item LiveOut[$i$] which instead contains the variables live \textit{after} the execution of $i$;
  \item Def[$i$] are the variables created by the instruction, in our language this set is either a singleton or the empty set as we can only define at most one variable per instruction;
  \item Use[$i$] are the variables used as arguments by the instruction, at most two in our language since that is the maximum arity of our expressions;
\end{itemize}

Since in our language instructions can be of three different kinds we define different dataflow equations for each instruction type.

We start by defining the dataflow equations for the ALU-instructions:
\begin{align*}
  \text{LiveIn}[i] &= \text{Use}[i] \cup (\text{LiveOut}[i] \setminus \text{Def}[i]) \\
  \text{LiveOut}[i] &= \bigcup \limits_{j \in \text{Succ[i]}} \text{LiveIn}[j]
\end{align*}

Since the computation is done backwards we start with the LiveOut[$i$] set, which must contain the variables that are required by the successors of $i$. The
LiveIn[$i$] set must instead contain the variables required by $i$, identified by Use[$i$] and the variables that are required by the next instructions, identified by LiveOut[$i$] $\setminus$ Def[$i$].
Here there are some things to note, first of all that the set LiveOut[$i_n$] = $\emptyset$ where $i_n$ is the last instruction of the program, since no variables are live after the program ends, we start from this assumption at the beginning of the procedure.

For jump instructions instead, the dataflow equations are similar, we just simplify the first dataflow equation by removing Def[$i$] as jump instructions never define variables.

Now, because of the different semantics of $\phi$-instructions we must define their dataflow equations separately. In particular, we define the liveness information of the whole section $\Phi$, instead of for single $\phi$-instructions. This is because, as was explained in \Cref{sec:ssa}, these instructions are executed parallelly.
We compute:
\begin{align*}
  \text{LiveIn}[\Phi] &= \text{PhiDefs}[\Phi] \cup \text{LiveOut}[\Phi] \\
  \text{LiveOut}[\Phi] &= \bigcup \limits_{j \in \text{Succ}[\Phi]} \text{LiveIn}[j]
\end{align*}

Where PhiDefs[$\Phi$] contains the left-hand sides of the $\phi$-instructions.

Starting again with LiveOut[$\Phi$], we define it as the union of the variables that are live before the first ALU-instruction of the block. LiveIn[$\Phi$] is instead the set of variables that are defined in the $\phi$-instructions together with the variables that are required after the $\phi$-instructions, intuitively the variables in PhiDefs[$\Phi$] are already defined before reaching the start of the block since their execution happens during the jump.

\section{Register Assignment}
\label{sec:ra}

% As explained in \Cref{subsec:ssara} register assignment is considered equivalent to the task of graph coloring of the interference graph. Before starting with the coloring let's recall \Cref{def:ig}. A chordal graph is a graph for which there exists a \gls{peo}, that is, there exists a simplicial node such that, if we remove that node the graph is still chordal. The \gls{peo} is then the ordering in which those nodes are removed.

In order to perform \gls{ra} we use the approach presented by Hack-Grund-Goos~\cite{HGG:2006:RA-SSA}. We detail the implementation in different sections:
\begin{itemize}
  \item In \Cref{subsec:ig}, we use the liveness information, obtained from the liveness analysis step, to build the interference graph;
  \item In \Cref{subsec:peo}, we iteratively remove the simplicial nodes from the interference graph obtaining a \gls{peo};
  \item Finally, in \Cref{subsec:coloring}, we reinsert the nodes of the \gls{peo} in \textit{reverse} order into the interference graph. While doing that, we assign them a color that is not already taken by their neighborhoods;
\end{itemize}

\subsection{Building the Interference Graph}
\label{subsec:ig}

After extracting the liveness information from our \gls{cfg}, we use it to build the interference graph.
The reasoning behind the creation of the interference graph is straightforward. We go through every live set computed during the liveness analysis, and we insert it as a clique in the interference graph. This follows intuitively by the fact that, if two variables are live in the same instruction, then they interfere, as stated in \Cref{def:ig}.

The function used to populate the interference graph is the following:

\begin{lstlisting}[style=Rocq]
Definition get_ig (pi : ProgramInfo.dict) : InterfGraph.dict :=
  fold_left
    (fun g l =>
      match ProgramInfo.get pi l with
      | Some (BlockInfo iis) => ig_insert_instinfos g iis
      | None => g
      end)
    (ProgramInfo.keys pi)
    InterfGraph.empty.
\end{lstlisting}

At line 5, we call \texttt{ig\_insert\_instinfos}. This function, given an interference graph and a list of live sets, returns a graph where a clique is added for each live set of the list.

\subsection{Obtaining a \glsentrylong{peo}}
\label{subsec:peo}

Recalling \Cref{def:simplicial} we implement a function that determines whether a node of a graph is simplicial:

\begin{lstlisting}[style=Rocq]
Definition is_simplicialb (g : InterfGraph.dict) (r : reg) : bool :=
  regs_mem r (InterfGraph.keys g) &&
  is_cliqueb g (InterfGraph.get g r ).
\end{lstlisting}

At line 2, we check that \texttt r is actually part of the graph using \texttt{regs\_mem}. At line 3, we check whether the neighborhood of \texttt r is a clique, using \texttt{is\_cliqueb}.
We now define a function that lets us find any simplicial node in a graph:

\begin{lstlisting}[style=Rocq]
Definition find_next (g : InterfGraph.dict) : option reg :=
  find (is_simplicialb g) (InterfGraph.keys g).
\end{lstlisting}

Here, we find the first node of the graph that respects the \texttt{(is\_simplicialb g)} condition. Note that the output node is an optional value, since the function may also receive a non-chordal graph which may not have a simplicial node. Because of that, we need to propagate the option wrapper to all the return values of the subsequent functions.
As shown in \Cref{cha:extraction}, upon receiving a \texttt{None} constructor, we throw an exception, signaling to the user that the interference graph (and likely even the initial \gls{jair} program) is ill-formed.

Now we define one step of the \texttt{eliminate} function, namely \texttt{eliminate\_step}:

\begin{lstlisting}[style=Rocq]
Definition eliminate_step
  (g : InterfGraph.dict) : option (reg * InterfGraph.dict):=
  match find_next g with
  | Some next =>
    Some (next, ig_remove_node g next)
  | None => None
  end.
\end{lstlisting}

This function takes an interference graph (which is chordal), and finds a simplicial node, removes it, and returns both the removed node and the resulting graph.
Note that, as we prove in \Cref{lem:inv-elim-2}, if the input graph is chordal and non-empty, the function will never return the \texttt{None} constructor. Furthermore, the resulting graph will also be chordal.

Now we define the \texttt{eliminate} function, that is, the function for obtaining a \gls{peo}:

\begin{lstlisting}[style=Rocq]
Function eliminate
  (g : InterfGraph.dict) {measure InterfGraph.size g} : list reg :=
  match eliminate_step g with
  | Some (next, g') => next :: (eliminate g')
  | None => nil
  end.
\end{lstlisting}

The definition of the function is straightforward. We take an interference graph, and we remove the simplicial nodes until we obtain the empty graph. At each recursive call, we append the removed node at the end of the list.
As was mentioned in \Cref{subsec:funterm}, Rocq has trouble automatically verifying the termination of this function. This forces us to introduce the annotation \texttt{{measure InterfGraph.size g}}. This tells Rocq that the decreasing property lies in the size of the interference graph. We will also prove this in \Cref{thm:term-elim}.

\subsection{Coloring}
\label{subsec:coloring}

Before talking about the algorithm~\cite{HGG:2006:RA-SSA} mentioned at the beginning of \Cref{sec:ra}, we make the following assumptions:

\begin{assumption}\label{ass:ig1}
The interference graph $G$ obtained from the liveness analysis is chordal, which we know from \Cref{thm:chordal-chromatic}.
\end{assumption}

\begin{assumption}\label{ass:ig2}
The chromatic number $\omega(G)$ is less than or equal to the number of available registers $k$. That is, we assume that spilling already happened, making the graph $k$-colorable.
\end{assumption}

Now consider a single iteration of the algorithm~\cite{HGG:2006:RA-SSA} mentioned at the beginning of \Cref{sec:ra}. Given \Cref{ass:ig1}, we are able to find a simplicial node $u$. This means that the neighborhood of $u$ forms a clique. Even if not every element of the neighborhood is colored, the colored neighbors still form a clique (a subset of a clique is still a clique). The next step consists in finding a color that is not used by the colored neighbors, which is always possible, because of \Cref{ass:ig2}.

Given this intuition, we start with the definition of the \texttt{Coloring} object. This is a map from virtual registers to physical registers and will be the output of the coloring phase.

At each position of the \textit{reversed} \gls{peo}, we find a node whose colored neighborhood forms a clique. Because of that, in order to assign this node a new color, we pick a random one from the complement of the neighborhood colors. We define a function specifically for this purpose.

\begin{lstlisting}[style=Rocq]
Definition preg_compl (colors : set preg) : option preg :=
  match IRPreg.regs_diff preg_allowed colors with
  | nil => None
  | c :: _ => Some c
  end.
\end{lstlisting}

Here, given a set of registers we calculate its difference with the set of registers in \texttt{preg\_allowed}. The variable \texttt{preg\_allowed} contains the registers that can be used during the coloring. Note that the result of this function is optional, since if we run out of registers, we are not able to extract a color. However, this cannot happen due to \Cref{ass:ig2}.

Now we use the function we just defined to find the color of a node given its interference graph and the current coloring:

\begin{lstlisting}[style=Rocq]
Definition get_color
  (v : IRVreg.reg)
  (g : InterfGraph.dict)
  (c : Coloring.dict)
  : option IRPreg.reg :=
  let nbors := InterfGraph.get g v in
  let used  := map (Coloring.get c) nbors in
  preg_compl used.
\end{lstlisting}

At line 6, we find the neighborhood of the node. At line 7, we map each neighbor to its assigned color and, at line 8, we extract an unused color.

We introduce the function to obtain the complete coloring of the interference graph, the procedure is the same as explained at the beginning of \Cref{subsec:coloring}.

\begin{lstlisting}[style=Rocq]
Fixpoint get_coloring_aux
  (peo : list IRVreg.reg)
  (g : IFG.dict)
  (c : Coloring.dict)
  : option Coloring.dict :=
  match peo with
  | nil => Some c
  | v :: peo =>
    match get_color v g c with
    | Some p => get_coloring_aux peo g (Coloring.update c v p)
    | None => None
    end
  end.
\end{lstlisting}

At line 6, we extract the head of the \gls{peo}. We then color it with the \texttt{get\_color} function. If a color is found, we proceed with a recursive call on the rest of the \gls{peo}. Otherwise, we return the \texttt{None} constructor.
The \texttt{get\_coloring\_aux} function requires some specific parameters to be passed initially. In particular, it must be provided with a \textit{reversed}\texttt{peo}, and an empty coloring. We define these parameters in the wrapper \texttt{get\_coloring}.

\begin{lstlisting}[style=Rocq]
Definition get_coloring
  (peo : list IRVreg.reg)
  (g : InterfGraph.dict)
  : option Coloring.dict :=
  get_coloring_aux (rev peo) g Coloring.empty.
\end{lstlisting}

After obtaining \texttt{Coloring} map, we visit the \gls{cfg}, and we color each register we encounter. We do that with the \texttt{color\_program} function.

\begin{lstlisting}[style=Rocq]
CoFixpoint color_program (c : Coloring.dict) (p : IRVreg.program)
  : IRPreg.program :=
  match p with
  | IRVreg.Block l ps is j =>
    IRPreg.Block l

    (* Color phi-instructions *)
    (map (color_phi c) ps)

    (* Color ALU-instructions *)
    (map (color_inst c) is)

    (match j with
    (* Color the conditional jump and the successor blocks *)
    | IRVreg.CondJump c' r v b1 b2 =>
      IRPreg.CondJump c'
      (Coloring.get c r)
      (color_val c v)
      (color_program c b1)
      (color_program c b2)

    (* Color the unconditional jump and the successor block *)
    | IRVreg.Jump b => IRPreg.Jump (color_program c b)

    (* Color the return register *)
    | IRVreg.Ret r => IRPreg.Ret (Coloring.get c r)
    end)
  end.
\end{lstlisting}

This function takes a coloring, a program using virtual registers, and returns a program using physical registers. At line 8, we color each $\phi$-instruction with function \texttt{color\_phi}. We do the same for the ALU-instructions at line 10. Instead, at line 13, we match with the different cases of jump instructions, color their arguments, and then color the successors with a recursive call.

In the \texttt{color\_program} function, lies the reason why we introduced the \texttt{UNASSIGNED} constructor in the physical register type. Assume, in fact, that instead of using that constructor, we used an \texttt{option} to wrap the return type of the \texttt{Coloring} map. Because of that, \texttt{Coloring} would become a partial map, rather than a total map. This would force us to return optional values in \texttt{color\_phi}, \texttt{color\_inst}, and even \texttt{color\_program}. However, because the latter is a \texttt{CoFixpoint} function, as defined in \Cref{subsec:coind}, its return type must be \texttt{CoInductive}. However, this is not the case for the \texttt{option} type.

\section{SSA Destruction}
\label{sec:destruct}

The final phase before emitting assembly is SSA destruction. SSA destruction translate $\phi$-instructions into register copies. This is required since $\phi$-instructions are not implemented in common hardware architectures.
To convert each sequence of $\phi$-instructions, we use the algorithm proposed by Rideau-Leroy~\cite{Rideau-Leroy-regalloc}, implemented in the \texttt{pmove} function. This function has signature \texttt{moves $\to$ moves}, where \texttt{moves} is a list of pairs of registers \texttt r and \texttt{r'}, the sources and destinations of the copy respectively.
Because we do not introduce a proof of termination for the algorithm, we pass \texttt{fuel} as an argument of the subsequent functions.

We then define \texttt{succ\_to\_insts}. This function takes the current block label, a successor block, and \texttt{fuel}. Then, it returns a sequence of register copies semantically equivalent to the ones specified by the $\phi$-instructions of the successor.

\begin{lstlisting}[style=Rocq]
Definition succ_to_insts (curr : lbl) (succ : block) (fuel : nat)
  : list inst :=
  let ms := phis_to_moves curr (get_phis succ) in
  let ms := pmove ms fuel in
  moves_to_insts ms.
\end{lstlisting}

Here, at line 3, we convert the $\phi$-instructions of the successor into simple moves. At line 4, we apply the translation algorithm passing the \texttt{fuel} parameter and the moves we just computed. At the end of the function, we return the moves as \gls{jair} instructions.
The following, is the function for performing \gls{ssa} destruction of the \gls{cfg}.

\begin{lstlisting}[style=Rocq]
Definition ssa_destruct (fuel : nat) (b : block) :=
  let cofix ssa_destruct_aux (curr: lbl) (b : block) : block :=
    match b with
    | Block l ps is j =>
      match j with
      | CondJump c r v b1 b2 =>
        Block l [] is (CondJump c r v
          (Block
            (Point1 (nat_of_lbl l)) [] (succ_to_insts l b1 fuel)
            (Jump (ssa_destruct_aux l b1)))
          (Block (Point2 (nat_of_lbl l)) [] (succ_to_insts l b2 fuel)
            (Jump (ssa_destruct_aux l b2))))
      | Jump b' =>
        Block l [] (is ++ (succ_to_insts l b' fuel))
          (Jump (ssa_destruct_aux l b'))
      | Ret r =>
        Block l [] is (Ret r)
      end
    end
  in
  ssa_destruct_aux (get_lbl b) b.
\end{lstlisting}

We identify three different cases when performing the destruction, one for each constructor of the jump instruction. At line 6, we handle the case of a \texttt{CondJump}. Here, we need to translate two different parallel moves, one for each of the two successors. In this case, we cannot append the two different parallel move translations at the end of the current block. If we did, they would conflict with each other. Because of that, we need to create two new basic blocks, namely \texttt{(Point1 l)} and \texttt{(Point2 l)}. Where \texttt l is the label of the current block. Each of the two new blocks contain the result of \texttt{succ\_to\_insts} and a jump instruction to the successor.

The second case, at line 13, is the unconditional \texttt{Jump}. The translation simply consists in appending the result of \texttt{succ\_to\_insts} at the end of the instructions of the current block. Maintaining the unconditional jump at the end of the block.

The last case, at line 14, handles the \texttt{Ret} instruction. This case does not require any translation, as it is the last instruction of a program.


% We mentioned in \Cref{sec:ssa} that a sequence of $\phi$-instructions behaves as a parallel copy, our goal is to find a sequence of register instruction that preserves this property. The translation is straightforward for some cases, take the following example, where $x \to y$ represents moving the content of register $x$ to register $y$:
% \[
%   r_1 \to r_2, r_2 \to r_3, r_3 \to r_4
% \]
% Solving this problem basically consists in reordering the moves so that no value is overwritten, the solution for this case is:
% \[
%   r_3 \to r_4, r_2 \to r_3, r_1 \to r_2
% \]
% Unfortunately some other cases are non-trivial, namely those who contain loops such as:
% \[
%   r_1 \to r_2, r_2 \to r_3, r_3 \to r_1
% \]
% There are two ways of solving these cases, we could try to find a sequence of swap operations such that at the end of the computation each value ends up in the intended register, this first option could be implemented using a swap instruction (like \texttt{xchg}) if supported by the architecture, or by using the XOR operator. Another option would be to reserve a special register \texttt{tmp} to perform the swaps. In our implementation we use an already verified algorithm~\cite{Rideau-Serpette-Leroy-parmov} that uses temporary variables. Given the previous example we would end up with the following result:
% \[
%   r_1 \to \texttt{tmp}, r_3 \to r_1, r_2 \to r_3, \texttt{tmp} \to r_2
% \]
% Now that we are able to compile a parallel move we visit the \gls{cfg} while translating one section of $\phi$-instructions at a time.

%TODO: MOVE THIS IN BACKGROUND
In \Cref{fig:destruct} we see an example of SSA destruction in action. The branch instruction at the end of $B_1$ introduces two additional blocks, $B_{1.1}$ and $B_{2.2}$. Additionally, the original $\phi$-instruction highlighted in block $B_1$ is translated into the instructions highlighted in blocks $B_{1.2}$ and \textbf{start}. Instead, in block $B_{1.1}$ no register copy is required, as the return argument in $B_2$ is already in $r_2$.

\begin{figure}[ht]
\begin{minipage}{0.45\textwidth}
\centering
  \begin{tikzpicture}[
      node distance=10mm,
      every node/.style={draw, align=left, inner sep=4pt},
      >={Stealth}
    ]
    \node (entry)   at (0, 0)   [draw, label=above:\textbf{start}] {$r_0 \leftarrow 0$};
    \node (loop)    at (0, -2)  [draw, label=left:$B_1$] {\colorbox{yellow!40}{$r_1 \leftarrow \phi(r_0, r_2)$} \\ $r_2 \leftarrow r_1 + 1$};
    \node (end)     at (0, -4)  [draw, label=left:$B_2$] {ret $r_2$};
    \draw[->] (entry) -- (loop);
    \draw[->] ([xshift=10pt]loop.south) to[out=315, in=45, looseness=8] ([xshift=10pt]loop.north);
    \draw[->] (loop) -- (end);
    \end{tikzpicture}
\end{minipage}
\hfill
\begin{minipage}{0.45\textwidth}
  \centering
  \begin{tikzpicture}[
      node distance=10mm,
      every node/.style={draw, align=left, inner sep=4pt},
      >={Stealth}
    ]
    \node (entry)   at (0, 0)     [draw, label=above:\textbf{start}] {$r_0 \leftarrow 0$ \\ \colorbox{yellow!40}{$r_1 \leftarrow r_0$}};
    \node (loop)    at (0, -2)    [draw, label=left:$B_1$] {$r_2 \leftarrow r_1 + 1$};
    \node (loop1)   at (-1.5, -4) [draw, label=left:$B_{1.1}$] {nop};
    \node (loop2)   at (1.5, -4)  [draw, label=left:$B_{1.2}$] {\colorbox{yellow!40}{$r_1 \leftarrow r_2$}};
    \node (end)     at (0, -6)    [draw, label=left:$B_2$] {ret $r_2$};
    \draw[->] (entry) -- (loop);
    \draw[->] (loop) -- (loop1);
    \draw[->] (loop) -- (loop2);
    \draw[->] (loop2.south) to[out=315, in=45, looseness=3] ([xshift=10pt]loop.north);
    \draw[->] (loop1) -- (end);
    \end{tikzpicture}
\end{minipage}
  \caption{Left: \gls{cfg} before \gls{ssa} destruction. Right: the same \gls{cfg} after \gls{ssa} destruction.}
  \label{fig:destruct}
\end{figure}
  % \caption{The \gls{cfg} before and after \gls{ssa} destruction. Note that blocks $B_{1.1}$ and $B_{2.2}$ have been added, and that the $\phi$-instruction was translated into copies (marked yellow).}


% \chapter{Alberi di classificazione} \label{chapter alberi}
% % Errors

\chapter{Verification}
\label{cha:verification}

% \chapter*{Conclusioni}
% \markboth{\MakeUppercase{Conclusioni}}{}
% \addcontentsline{toc}{chapter}{Conclusioni}
% \input{capitoli/conclusione}

\appendix
% \chapter{Scelta dei parametri}
% \input{capitoli/appendiceA}
% \chapter{Barca}
% \input{capitoli/appendiceB}

\backmatter

\printbibliography
\addcontentsline{toc}{chapter}{Bibliography}

\listoffigures
\addcontentsline{toc}{chapter}{List of Figures}

\listoftables
\addcontentsline{toc}{chapter}{List of Tables}

\printindex
\addcontentsline{toc}{chapter}{Indice}


\end{document}