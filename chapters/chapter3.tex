% Errors

\chapter{Implementation}
\label{cha:implementation}

In this chapter, we provide an overview of the design and implementation choices that guided the development of the register assignment pipeline and the underlying language \gls{jair}. The implementation is structured in a modular way, beginning with the definition of the syntax and semantics of the language. We then detail the steps of register assignment, namely liveness analysis, graph coloring, and SSA destruction.

\section{\gls{jair}}
\label{sec:jair}

The primary goal in defining the syntax of \gls{jair} is to leverage Rocq’s type system to rule out as many ill-formed SSA programs as possible, as discussed in \Cref{sec:ssa}. We begin by describing how registers are represented in our language.

\subsection{Registers}

Our implementation uses two kinds of registers: virtual registers in the earlier phases of register assignment, and physical registers in the final phase. To support both use cases, we define \texttt{IR} module, which defines the syntax of our intermediate representation generically over these two types of registers.

The module requires the following parameters:
\begin{itemize}
    \item The register type \texttt{reg}, the set of all the registers we can choose from;
    \item The boolean equality function for the register type, namely \texttt{reg\_eqb};
    \item A proof for the decidability of the logical equality for the register type, namely \texttt{reg\_eq\_dec};
\end{itemize}

For virtual registers, we use the set of natural numbers, which, although bounded by Rocq's runtime limits, can be treated as unbounded for our purposes.

\begin{lstlisting}[style=Rocq]
Definition vreg := nat.
\end{lstlisting}

For physical registers, instead, we define a finite set that corresponds to the 64-bit general-purpose registers of the x86-64 architecture.

\begin{lstlisting}[style=Rocq]
Inductive preg : Type :=
  | RAX
  | RBX
  | RCX
  (* ... *)
  | UNASSIGNED
.
\end{lstlisting}

We also include the \texttt{UNASSIGNED} register, this register is used as a default value for the coloring, removing the need of returning an optional value everywhere. Later we will prove that the coloring function will never return this constructor.

Throughout the implementation, every register is understood to be either virtual or physical depending on the phase of the compilation pipeline.

\subsection{Labels}

Initially, we opted for a label-less representation of basic blocks.
Later in the project we saw the necessity of introducing them for this specific reason: the $\phi$-instructions require labels to identify the incoming control flow since predecessors of a basic blocks are not ordered. Without labels a $\phi$-instruction would not know which one of the predecessors each of its arguments is bound to.

An alternative solution would be (given the single definition policy of SSA) to go back in the control flow to find which of the arguments of the $\phi$-instruction is defined. This approach proved too complicated, so we adopted labels, at the cost of introducing possible inconsistencies in the representation that we will later discuss in \Cref{subsec:limitations}.

\begin{lstlisting}[style=Rocq]
Inductive lbl : Type :=
  | Normal : nat -> lbl
  | Point1 : nat -> lbl
  | Point2 : nat -> lbl.
\end{lstlisting}

The label type is defined with three constructors. The reason is explained more in detail in \Cref{sec:destruct}, but for now let's just say that the SSA destruction phase introduces additional basic blocks (at most two for each preexisting basic block). With the constructors \texttt{Point1} and \texttt{Point2} we identify these additional basic blocks in order to avoid collisions, whereas with the \texttt{Normal} constructor we identify the blocks that already existed before the destruction phase.

\subsection{ALU Instructions}

In order to perform computations, our intermediate representation must provide arithmetical and logical instructions.
We start by defining the \texttt{val} type which represents an operand of these instructions. It can be either an immediate integer, a register, or a pointer to a memory location (represented by a natural number):

\begin{lstlisting}[style=Rocq]
Inductive val : Type :=
  | Imm (x : Z)
  | Reg (r : reg)
  | Ptr (p : nat).
\end{lstlisting}

We then define the type of expressions \texttt{expr}. These include register copies, memory loads, and arithmetical and logical operations. Expression depth is restricted to avoid unnecessary complexity and to preserve the linearity of the data structure.

\begin{lstlisting}[style=Rocq]
Inductive expr : Type :=
  | Val : val -> expr
  | Load : val -> expr
  | Add : reg -> val -> expr
  | Sub : reg -> val -> expr
  (* ... *)
.
\end{lstlisting}

As can be seen from the previous snippet, binary expressions always use a register as the first operand, preventing expressions that involve only constants since, in that case, the result could be computed during a previous step of constant folding. Another thing to note is that, for the sake of register assignment, a differentiation between unary, binary and $n$-ary expressions is unnecessary, as we are only concerned with the identities of the operands of an expression (the registers) and not the operators.

Finally, the instruction type \texttt{inst} reflects the core operations in our language. Instructions either define a register by assigning to it the result of an expression, or store a value into memory:

\begin{lstlisting}[style=Rocq]
Inductive inst : Type :=
  | Def (r : reg) (e : expr)
  | Store (v : val) (r : reg).
\end{lstlisting}

The \texttt{Store} instruction is treated specially because it does not produce a result that can be assigned to a register. Therefore, it cannot be expressed as a \texttt{Def}.

\subsection{$\phi$-Instructions}

In \Cref{sec:ssa} we defined a section of $\phi$-instructions as a parallel move based on the predecessor in the control flow.
We need an implementation that preserves this definition, to do so we start by defining the type of an argument of a $\phi$-instruction, namely \texttt{phi\_arg}.

\begin{lstlisting}[style=Rocq]
Definition phi_arg : Type := (reg * lbl).
\end{lstlisting}

Semantically the \texttt{reg} instance is the source of the copy if the control flow comes from the \texttt{lbl} instance.

Finally, we define $\phi$-instruction type, this is simply an assignment of one of the possible \texttt{phi\_arg}s to a destination \texttt{reg}.

\begin{lstlisting}[style=Rocq]
Inductive phi : Type :=
| Phi : reg -> list phi_arg -> phi.
\end{lstlisting}

% Here, each \texttt{phi\_arg} pairs a register with a label identifying the originating block. Initially, we experimented with a label-free control flow representation that used direct pointers to blocks. However, as we will explain later, this made the semantics unnecessarily difficult to define, leading us to prefer label-based control flow.

\subsection{Blocks and Jump Instructions}

ALU instructions and $\phi$-instructions are defined separately enforcing the fact that in basic blocks the $\phi$-instructions appear first, followed by ALU instructions and finally by a jump instruction.
We define the jump instruction type in mutual recursion with the block type.

\begin{lstlisting}[style=Rocq]
CoInductive block : Type :=
  | Block (l : lbl) (ps : list phi) (is : list inst) (j : jinst)
with jinst : Type :=
  | CondJump : cond -> reg -> val -> block -> block -> jinst
  | Jump : block -> jinst
  | Ret : reg -> jinst.
\end{lstlisting}

Control flow is encoded directly using block references instead of labels, preventing accidental jumps to nonexistent blocks. This definition naturally forms a control-flow graph, where blocks are nodes and jump instructions define the edges.

Finally, a \gls{cfg} as we described in \Cref{subsec:cfg} is a set of blocks where we identify a start block and a control flow relation, it is sufficient to provide the following definition:

\begin{lstlisting}[style=Rocq]
Definition program : Type := block.
\end{lstlisting}

Where the starting point of the \gls{cfg} is the first block we encounter during a visit and the blocks set of the \gls{cfg} is the set of blocks reachable from the first block.

Now that we covered the main parts of the \gls{jair} implementation, we demonstrate the capabilities of our representation in \Cref{fig:example-jair} with an example of the Fibonacci function along with the same example in a general SSA form representation.

\begin{figure}[ht]
\centering
\begin{minipage}{0.65\textwidth}
\begin{lstlisting}[style=Rocq]
Definition example_block_3 : block :=
  Block (Normal 3) [] [] (ret r(6)).

CoFixpoint example_block_2 : block :=
  Block (Normal 2) [
    r(3) <- phi [(0, Normal 1); (4, Normal 2)];
    r(4) <- phi [(1, Normal 1); (6, Normal 2)];
    r(5) <- phi [(2, Normal 1); (7, Normal 2)]
  ] [
    r(6) <- r(4) + r(3);
    r(7) <- r(5) - i(1)
  ] (
    if r(7) = i(1)
    then example_block_3
    else example_block_2
  ).

Definition example_block_1 : block :=
  Block (Normal 1) [] [
    r(0) <- i(0);  (* Second temp *)
    r(1) <- i(1);  (* First temp *)
    r(2) <- i(12)  (* Iterator *)
  ] (
    Jump example_block_2
  ).
\end{lstlisting}
\end{minipage}
\hfill
\begin{minipage}{0.30\textwidth}
\centering
\begin{tikzpicture}[
    node distance=10mm,
    every node/.style={draw, align=left, inner sep=4pt},
    >={Stealth}
]
  \node (entry)   at (0, 0)   {$r_0 \leftarrow 0$ \\ $r_1 \leftarrow 1$ \\ $r_2 \leftarrow 12$};
  \node (loop)    at (0, -3)  {$r_3 \leftarrow \phi(r_0, r_4)$ \\ $r_4 \leftarrow \phi(r_1, r_6)$ \\ $r_5 \leftarrow \phi(r_2, r_7)$ \\ $r_6 \leftarrow r_4 + r_3$ \\ $r_7 \leftarrow r_5 - 1$};
  \node (end)     at (0, -5.5)  {ret $r_6$};
  \draw[->] (entry) -- (loop);
  \draw[->] ([xshift=10pt]loop.south) to[out=315, in=45, looseness=4] ([xshift=10pt]loop.north);
  \draw[->] (loop) -- (end);
\end{tikzpicture}
\end{minipage}
\caption{Code for the Fibonacci function in \gls{jair} along with the same code in a general SSA form representation}
\label{fig:example-jair}
\end{figure}

\subsection{Limitations}
\label{subsec:limitations}

Despite our efforts to use Rocq's type system to rule out incorrect programs, certain ill-formed programs cannot be excluded purely by the syntax, we now provide examples of such cases.

A fundamental problem is that the two core policies of an SSA form representation are not enforced, namely multiple assignments to the same register and the use of undefined registers are still possible:

\begin{lstlisting}[style=Rocq]
Definition double_assignment : block :=
  Block (Normal 0) [] [
    r(1) <- i(0);
    r(1) <- i(1)
  ]
  (ret r(1)).

Definition undefined_variable : block :=
  Block (Normal 0) [] [r(1) <- r(0)] (ret r(1)).
\end{lstlisting}

Similarly, uniqueness of labels is not enforced, which can lead to conflicting block labels, as shown in the following example:

\begin{lstlisting}[style=Rocq]
CoFixpoint double_lbl_1 : block :=
  Block (Normal 0) [] [] (Jump double_lbl_2)
with double_lbl_2 : block :=
  Block (Normal 0) [] [] (Jump double_lbl_1).
\end{lstlisting}

Finally, $\phi$-instructions may include inconsistent or invalid arguments. For example, the number of the predecessors listed in a $\phi$-instruction may not match the actual number of predecessors:

\begin{lstlisting}[style=Rocq]
Definition ill_formed_phi_2 : block :=
  Block (Normal 1) [
    r(1) <- phi [(1, 5); (2, 6); (3; 7)]] [] (ret r(0)).
Definition ill_formed_phi_1 : block :=
  Block (Normal 0) [] [] (Jump ill_formed_phi_2).
\end{lstlisting}

Although these issues cannot be resolved by the type system alone, they can be caught during a previous semantic analysis phase.

\section{A \gls{jair} Interpreter}
\label{sec:jair-int}

We now implement a small-step interpreter for our intermediate representation, we do this for two reasons.
The first one is to clearly state the semantics of \gls{jair}. The second one is that, in order to test register assignment, we will compare the result of a program executed on the interpreter virtual machine with the result of the same program executed on bare hardware, if the two results match this will suggest that the semantics of the program are preserved even after register assignment.
The definition of the virtual machine is straightforward as its components are just the register file, implemented as a map from registers to integers, and the memory, implemented as list of integers.

\begin{lstlisting}[style=Rocq]
Inductive vm : Type :=
  | Vm : (reg -> Z) -> list Z -> vm.
\end{lstlisting}

After defining the virtual machine, we define the primitives to interact with it. We begin with operations on the register file.

The function to read a value from a register is defined as follows:

\begin{lstlisting}[style=Rocq]
Definition get_reg (m : vm) (r : reg) : cell :=
  match m with
  | Vm regs _ => regs r
  end.
\end{lstlisting}

This function takes a virtual machine state \texttt m and a register \texttt r, and returns the content of the register  by performing $\beta$-reduction on the register file map.

The corresponding function to write a value into a register is defined as:

\begin{lstlisting}[style=Rocq]
Definition set_reg (m : vm) (r : reg) (c : cell) : vm :=
  match m with
  | Vm regs cells =>
    Vm (fun r' => if r' =? r then c else regs r') cells
  end.
\end{lstlisting}

Here, we take the current register file map, and we wrap it with a lambda function, if the argument of that function matches register \texttt r, we return the value \texttt c, otherwise we perform $\beta$-reduction with the previous version of the register file map.

Now, we define analogous operations for memory access. In our implementation memory does not support random access, instead, we use a list that grows whenever a new value is stored, filling the previously uninitialized memory locations with zeroes. The function to retrieve the content of a memory location is defined as follows:

\begin{lstlisting}[style=Rocq]
Definition get_cell (m : vm) (i : nat) : cell :=
  let fix get_cell_aux (cells : list cell) (i : nat) : cell :=
    match cells, i with
    | nil, _ => Z0
    | c :: _, O => c
    | _ :: cs, S i' => get_cell_aux cs i'
    end
  in
  match m with
  | Vm _ cells => get_cell_aux cells i
  end.
\end{lstlisting}

The recursive function considers three cases:
\begin{itemize}
  \item If the memory is empty, then the value we are trying to retrieve is zero;
  \item If the index is zero, we return the value of the current memory location;
  \item Otherwise, if we haven't yet reached the desired address, we perform a recursive call on the remaining memory;
\end{itemize}

Finally, the function for writing a value to a memory location is defined as:

\begin{lstlisting}[style=Rocq]
Definition set_cell (m : vm) (i : nat) (c : cell) : vm :=
  let fix set_cell_aux (cells : list cell) (i : nat) (c : cell) : list cell :=
    match cells, i with
    | nil, O => c :: nil
    | nil, S i' => Z0 :: (set_cell_aux nil i' c)
    | _ :: xs, O => c :: xs
    | x :: xs, S i' => x :: (set_cell_aux xs i' c)
    end
  in
  match m with
  | Vm regs cells => Vm regs (set_cell_aux cells i c)
  end.
\end{lstlisting}

The recursive function handles four cases:
\begin{itemize}
  \item If the memory is empty, but we reached the intended address, we expand the memory with one last cell where we store the content;
  \item If the memory is empty, and we haven't yet reached the desired address, we create an empty cell and make a recursive call;
  \item If we reached the intended address, we overwrite the current content with the new content;
  \item Finally, if we haven't yet reached the intended address, we perform a recursive call on the remaining memory;
\end{itemize}

The only remaining task is to define a function that simulates program execution, ideally we would want that function to have type \texttt{vm $\to$ program $\to$ vm} taking the initial state of the virtual machine and the program as inputs and returning the new state of the virtual machine.
Unfortunately, as we mentioned in \Cref{subsec:funterm}, Rocq does not allow for the definition of functions of which we cannot prove termination, and proving it for this function would entail solving the halting problem. We thus resort to using fuel based recursion for this case.

\begin{lstlisting}[style=Rocq]
Fixpoint run (m : vm) (p : program) (fuel : nat) : vm :=
  match p, fuel with
  | _, O => m
  | Block _ _ is j, S fuel' =>
    let m := run_insts m is in
    match j with
    | CondJump c r v b1 b2 =>
      if eval_cond m c r v then
        run (run_phis m p b1) b1 fuel'
      else
        run (run_phis m p b2) b2 fuel'
    | Jump b1 => run (run_phis m p b1) b1 fuel'
    | Ret r => set_reg m 0 (get_reg m r)
    end
  end.
\end{lstlisting}

The procedure works this way: if we are out of fuel we terminate, and we return the current state of the virtual machine, otherwise we proceed with the computation.
We run the body of instructions of the current block with the function \texttt{run\_insts}, yielding a new state of the virtual machine. We then treat each jump instruction differently, in the case of a conditional jump we evaluate the condition with the \texttt{eval\_cond} function, in the other two cases no evaluation is needed. Finally, before jumping to the next block we execute its $\phi$-instructions, to do so we call the \texttt{run\_phis} function passing as arguments the virtual machine, the current block and the successor block.
We continue with the computation until we either run out of fuel or we reach the \texttt{Ret} instruction, saving the result of the program to register zero.

\section{Liveness Analysis}
\label{sec:liveness}

In this phase, we compute the set of live variables at each program point. This step is crucial for determining which variables interfere and, therefore constitute an edge of the interference graph.

Generally a liveness analysis algorithm works this way:
we start from the last instruction of the program and go back to the start instruction in a post-order fashion, while we do that we compute the following sets for each instruction:
\begin{itemize}
  \item \texttt{live\_in[i]} which contains the variables that are live \textit{before} the execution of instruction \texttt i;
  \item \texttt{live\_out[i]} which instead contains the variables live \textit{after} the execution of \texttt i;
  \item \texttt{def[i]} are the variables created by the instruction, in our language this set is either a singleton or the empty set as we can only define at most one variable per instruction;
  \item \texttt{use[i]} are the variables used as arguments by the instruction, at most two in our language since that is the maximum arity of our expressions;
\end{itemize}

Since in our language instructions can be of three different kinds we define different dataflow equations for each instruction type.

We start by defining the dataflow equations for the ALU instructions:
\begin{align*}
  \texttt{live\_in[i]} &= \texttt{use[i]} \cup (\texttt{live\_out[i]} \setminus \texttt{def[i]}) \\
  \texttt{live\_out[i]} &= \bigcup \limits_{\texttt j \in \texttt{succ[i]}} \texttt{live\_in[j]}
\end{align*}

Since the computation is done backwards we start with the \texttt{live\_out[i]} set, which must contain the variables that are required by the successors of \texttt i. The
\texttt{live\_in[i]} set must instead contain the variables required by \texttt i, identified by \texttt{use[i]} and the variables that are required by the next instructions, identified by \texttt{live\_out[i]} $\setminus$ \texttt{def[i]}.
Here there are some things to note, first of all that the set $\texttt{live\_out[i\_n]} = \emptyset$ where \texttt{i\_n} is the last instruction of the program, since no variables are live after the program ends, we start from this assumption at the beginning of the procedure.

For jump instructions instead, the dataflow equations are similar, we just simplify the first dataflow equation by removing \texttt{def[i]} as jump instructions never define variables.

Now, because of the different semantics of $\phi$-instructions we must define their dataflow equations separately. In particular, we define the liveness information of the whole section instead of for single $\phi$-instructions, this is because, as was explained in \Cref{subsec:phi}, these instructions are executed parallelly.
We compute:
\begin{align*}
  \texttt{live\_in[ps]} &= \texttt{phi\_defs[ps]} \cup \texttt{live\_out[ps]} \\
  \texttt{live\_out[ps]} &= \bigcup \limits_{\texttt j \in \texttt{succ[ps]}} \texttt{live\_in[j]}
\end{align*}
where \texttt{phi\_defs[ps]} contains the left-hand sides of the $\phi$-instructions.

Starting again with \texttt{live\_out[ps]}, we define it as the union of the variables that are live before the first ALU instruction of the block. \texttt{live\_in[ps]} is instead the set of variables that are defined in the $\phi$-instructions together with the variables that are required after the $\phi$-instructions, intuitively the variables in \texttt{phi\_defs[ps]} are already defined before reaching the start of the block since their execution happens during the jump.

\section{Register Assignment}
\label{sec:ra}

As explained in \Cref{subsec:ssara} register assignment is considered equivalent to the task of graph coloring of the interference graph, but, before starting with the coloring let's recall \Cref{def:ig}. A chordal graph is a graph for which there exists a \gls{peo}, that is, there exists a simplicial node such that, if we remove that node the graph is still chordal and, the \gls{peo} is then the ordering in which those nodes are removed.
To perform the coloring we then proceed this way:
\begin{itemize}
  \item We use the liveness information to build the interference graph;
  \item We iteratively remove the simplicial nodes from the interference graph obtaining a \gls{peo};
  \item We reinsert the nodes of the \gls{peo} in \textit{reverse} order into the interference graph and, at the same time, we assign them a color that is not already taken by their (partially colored) neighborhoods;
\end{itemize}

We now proceed with the implementation of the steps that we just described.

\subsection{Building the Interference Graph}

Now that we extracted the liveness information from our \gls{cfg} we use it to build the interference graph.
The reasoning behind the creation of the interference graph is straightforward, we go through every live set computed in the previous step, and we insert a clique containing its elements. This follows intuitively by the fact that if two variables are live in the same instruction then they interfere as per \Cref{def:ig}.

The function used to populate the interference graph is the following:

\begin{lstlisting}[style=Rocq]
Definition get_ig (pi : ProgramInfo.dict) : InterfGraph.dict :=
  fold_left
    (fun g l =>
      match ProgramInfo.get pi l with
      | Some (BlockInfo iis) => ig_insert_instinfos g iis
      | None => g
      end)
    (ProgramInfo.keys pi)
    InterfGraph.empty.
\end{lstlisting}

Where \texttt{ig\_insert\_instinfos} is a function that given an interference graph and a list of live sets returns a graph where a clique containing the elements of the set is added for each element of the list.

\subsection{Obtaining a \glsentrylong{peo}}
\label{subsec:peo}

Recalling \Cref{def:simplicial} we implement a function that determines whether a node of a graph is simplicial:

\begin{lstlisting}[style=Rocq]
Definition is_simplicialb (g : InterfGraph.dict) (r : reg) : bool :=
  let nbors := InterfGraph.get g r in
  regs_mem r (InterfGraph.keys g) &&
  is_cliqueb g nbors.
\end{lstlisting}

In this case \texttt{InterfGraph.get} is simply the function to get the neighborhood of a node, \texttt{regs\_mem} is a boolean function to test membership of a node to a set and finally \texttt{is\_cliqueb} is a boolean function to test whether a set is a clique in a specific graph.
For the purpose of this project, let's ignore the complexity of the various functions as we are only concerned in building a framework that makes it easy to perform formal verification. Anyway, be reassured that all the functions run in polynomial time.

We now define a function that lets us find a simplicial node in the graph:

\begin{lstlisting}[style=Rocq]
Definition find_next (g : InterfGraph.dict) : option reg :=
  find (is_simplicialb g) (InterfGraph.keys g).
\end{lstlisting}

As explained in the second step of the coloring procedure, at each iteration we need to find a simplicial node, and that's exactly what \texttt{find\_next} does. Note that the output node is an optional value, since the function may also receive a non-chordal graph which may not have a simplicial node. This in reality could not be the case since we make the assumption that the interference graphs we receive are chordal. Unfortunately because of that we need to propagate the option wrapper to all the return values of the next functions.
In the extracted code, upon receiving a \texttt{None} constructor we will to throw an exception, signaling to the user that the interference graph (and likely even the initial \gls{jair} program) is malformed.

Now we proceed with the step function for the \texttt{eliminate} function:

\begin{lstlisting}[style=Rocq]
Definition eliminate_step
  (g : InterfGraph.dict) : option (reg * InterfGraph.dict):=
  match find_next g with
  | Some next =>
    Some (next, ig_remove_node g next)
  | None => None
  end.
\end{lstlisting}

It takes an interference graph which is chordal, finds a simplicial node, removes it, and returns both the removed node and the resulting graph.
Note that, if the precondition of the input graph being chordal is met, the function will never return the \texttt{None} constructor. Furthermore, the resulting graph will also be chordal.
We will talk more about the postconditions in \Cref{cha:verification}.

Finally, we can move into the \texttt{eliminate} function, that is, the function for obtaining a \gls{peo}:

\begin{lstlisting}[style=Rocq]
Function eliminate
  (g : InterfGraph.dict) {measure InterfGraph.size g} : list reg :=
  match eliminate_step g with
  | Some (next, g') => next :: (eliminate g')
  | None => nil
  end.
\end{lstlisting}

The definition of the function is straightforward, we take an interference graph, and we remove the simplicial nodes until it becomes the empty graph, constantly appending the removed node at the end of the list.
As was mentioned in \Cref{subsec:funterm} Rocq has trouble automatically verifying the termination of this function, because of that we add an annotation that tells it that the fixed point lies in the size of the interference graph. We will write a proof for this in \Cref{cha:verification}.

\subsection{Coloring}
\label{subsec:coloring}

The intuition behind the coloring algorithm is the following, first we make the following assumptions:

\begin{itemize}
  \item The interference graph $G$ obtained from the previous phase is chordal, which we know from \Cref{thm:chordal-chromatic};
  \item The chromatic number $\omega(G)$ is less than or equal to the number of available registers $k$, that is, we assume that spilling already happened, making the graph $k$-colorable;
\end{itemize}

Now take into consideration a single iteration of the algorithm mentioned at the beginning of \Cref{sec:ra}. The node we are about to color is simplicial, meaning that its neighborhood forms a clique. Even if not every element of the neighborhood is colored, the colored neighbors still form a clique (a subset of a clique is still a clique) meaning that the next step consists in finding a color that is not used by the colored neighbors, which is always possible since we assume that $\omega(G) \leq k$.

Given the intuition behind the algorithm, we start with the definition of the \texttt{Coloring} object as a dictionary from virtual registers to physical registers, this object will be the output of the coloring phase.

As we mentioned before, at each position of the \textit{reversed} \gls{peo} we find a node whose colored neighborhood forms a clique, because of that, in order to assign this node a new color we pick a random one from the complement of the neighborhood colors. We define a function specifically for this purpose.

\begin{lstlisting}[style=Rocq]
Definition preg_compl (colors : set preg) : option preg :=
  match IRPreg.regs_diff preg_allowed colors with
  | nil => None
  | c :: _ => Some c
  end.
\end{lstlisting}

Here, given a set of registers we calculate the difference with the set of registers \texttt{preg\_allowed}, and we extract an arbitrary element from it, note that the result of this function is optional since, if we run out of registers, we are not able to extract from the complement and ultimately perform register assignment. This, however, is an unexpected case solved by spilling.

Now we use the function we just defined to color a single node given its interference graph and the current coloring:

\begin{lstlisting}[style=Rocq]
Definition get_color
  (v : IRVreg.reg) (g : InterfGraph.dict) (c : Coloring.dict)
  : option IRPreg.reg :=
  let nbors := InterfGraph.get g v in
  let used := map (Coloring.get c) nbors in
  preg_compl used.
\end{lstlisting}

Finally, we introduce the function to obtain the complete coloring of the interference graph, the procedure is the same as explained at the beginning of \Cref{subsec:coloring}.

\begin{lstlisting}[style=Rocq]
Definition get_coloring (peo : list IRVreg.reg) (g : InterfGraph.dict)
  : option Coloring.dict :=
  let fix get_coloring_aux (peo : list IRVreg.reg) (c : Coloring.dict)
    : option Coloring.dict :=
    match peo with
    | nil => Some c
    | v :: peo =>
      match get_color v g c with
      | Some p =>
        let c := Coloring.update c v p in
        get_coloring_aux peo c
      | None => None
      end
    end
  in
  get_coloring_aux (rev peo) Coloring.empty.
\end{lstlisting}

\section{SSA Destruction}
\label{sec:destruct}

The final phase before emitting assembly is SSA destruction, which consists in the translation of $\phi$-instructions into moves. This final phase is required since $\phi$-instructions are not implemented in common architectures.
We mentioned in \Cref{subsec:phi} that a section of $\phi$-instructions behaves as a parallel copy, our goal is to find a sequence of instruction that preserves this property. The translation is straightforward for some cases, take the following example, where $x \to y$ represents moving the content of register $x$ to register $y$:
\[
  r_1 \to r_2, r_2 \to r_3, r_3 \to r_4
\]
Solving this problem basically consists in reordering the moves so that no value is overwritten, the solution for this case is:
\[
  r_3 \to r_4, r_2 \to r_3, r_1 \to r_2
\]
Unfortunately some other cases are non-trivial, namely those who contain loops such as:
\[
  r_1 \to r_2, r_2 \to r_3, r_3 \to r_1
\]
There are two ways of solving these cases, we could try to find a sequence of swap operations such that at the end of the computation each value ends up in the intended register, this first option could be implemented using a swap instruction (like \texttt{xchg}) if supported by the architecture, or by using the XOR operator. Another option would be to reserve a special register \texttt{tmp} to perform the swaps. In our implementation we use an already verified algorithm~\cite{Rideau-Serpette-Leroy-parmov} that uses temporary variables. Given the previous example we would end up with the following result:
\[
  r_1 \to \texttt{tmp}, r_3 \to r_1, r_2 \to r_3, \texttt{tmp} \to r_2
\]
Now that we are able to compile a parallel move we visit the \gls{cfg} while translating one section of $\phi$-instructions at a time.
We identify three different cases when performing the destruction, one for each jump instruction:
\begin{itemize}
  \item When visiting a basic block that ends with a \texttt{CondJump} we need to translate two different parallel moves, one for each of the two successors, since we cannot append two different parallel move translations into the current block, as they would conflict with each other, we need to create two new basic blocks, each of which contains the corresponding translation of the parallel move and then an unconditional jump to the successor. We also give them the same label as the current block but, this time, we use constructors \texttt{Point1} and \texttt{Point2} to avoid a collision with the current block, which uses the \texttt{Normal} constructor;
  \item If instead we encounter a jump instruction, the translation simply consists in appending the translated parallel move at the end of the instructions of the current block;
  \item Finally the return instruction does not require any translation, as it is the last instruction of a program;
\end{itemize}

In \Cref{fig:destruct} we can see an example of SSA destruction in action, in the visit first we encounter an unconditional jump, then a conditional jump and finally a return instruction.


\begin{figure}[h]
\centering
\begin{minipage}{0.45\textwidth}
\centering
  \begin{tikzpicture}[
      node distance=10mm,
      every node/.style={draw, align=left, inner sep=4pt},
      >={Stealth}
    ]
    \node (entry)   at (0, 0)   {$r_0 \leftarrow 0$};
    \node (loop)    at (0, -2)  {$r_1 \leftarrow \phi(r_0, r_2)$ \\ $r_2 \leftarrow r_1 + 1$};
    \node (end)     at (0, -4)  {ret $r_2$};
    \draw[->] (entry) -- (loop);
    \draw[->] ([xshift=10pt]loop.south) to[out=315, in=45, looseness=8] ([xshift=10pt]loop.north);
    \draw[->] (loop) -- (end);
    \end{tikzpicture}
\end{minipage}
\hfill
\begin{minipage}{0.45\textwidth}
\centering
  \begin{tikzpicture}[
      node distance=10mm,
      every node/.style={draw, align=left, inner sep=4pt},
      >={Stealth}
    ]
    \node (entry)   at (0, 0)     {$r_0 \leftarrow 0$ \\ $r_1 \leftarrow r_0$};
    \node (loop)    at (0, -2)    {$r_2 \leftarrow r_1 + 1$};
    \node (loop1)   at (-1.5, -4) {nop};
    \node (loop2)   at (1.5, -4)  {$r_1 \leftarrow r_2$};
    \node (end)     at (0, -6)    {ret $r_6$};
    \draw[->] (entry) -- (loop);
    \draw[->] (loop) -- (loop1);
    \draw[->] (loop) -- (loop2);
    \draw[->] (loop2.south) to[out=315, in=45, looseness=3] ([xshift=10pt]loop.north);
    \draw[->] (loop1) -- (end);
    \end{tikzpicture}
\end{minipage}
\caption{SSA destruction of a loop}
\label{fig:destruct}
\end{figure}

\todo{Add description of registers implementation either in \Cref{sec:jair} or \Cref{cha:verification}}

\section{Extraction}
\label{sec:extract}

After defining all the steps of our register assignment pipeline, we extract them into OCaml.
In particular, we extract the following functions:
\begin{itemize}
  \item \texttt{Vm.run} to interpret our programs;
  \item \texttt{LivenessInfo.analyze\_program} to perform liveness analysis;
  \item \texttt{InterfGraph.get\_ig} to obtain the interference graph;
  \item \texttt{Peo.eliminate} to obtain the \gls{peo};
  \item \texttt{Color.get\_coloring} to construct our coloring map;
  \item \texttt{Color.color\_program} to translate the program into one using physical registers;
  \item \texttt{Destruct.ssa\_destruct} to remove the $\phi$-instructions;
\end{itemize}

After extracting we compose the various steps of our pipeline into the \texttt{regassign} function:

\begin{lstlisting}[style=OCaml]
let regassign irvreg_program =
  let (programinfo, _) = analyze_program irvreg_program fuel_analyze in
  let interfgraph = get_ig programinfo in
  let peo = eliminate interfgraph in
  let coloring =
    match get_coloring peo interfgraph with
    | Some c -> c
    | None -> failwith "Not enough registers to complete assignment"
  in
  let irpreg_program = color_program coloring irvreg_program in
  ssa_destruct fuel_destruct irpreg_program
;;
\end{lstlisting}

This function accepts a program written in \gls{jair} and returns an equivalent program without $\phi$-instructions and using only x86-64 registers.
Note that, at line 8, we panic if we are not able to perform the coloring due to a high number of variables, this is required since we do not perform spilling.

Finally, with this representation, the generation of assembly code is straightforward, we simply perform a DFS on the \gls{cfg} and generate the assembly code for each basic block we visit.
In particular, for our implementation, we chose to emit assembly with the Netwide Assembler (NASM) syntax.

The following OCaml function does exactly so, writing the assembly code of a program into the specified output channel:

\begin{lstlisting}[style=OCaml]
let gen_irpreg_program out program =
  let visited = ref LblSet.empty in
  let rec gen_irpreg_program_aux out program =
    let IRPreg.Block (l, ps, is, j) = Lazy.force_val program in

    if not (LblSet.mem l !visited) then (
      visited := LblSet.add l !visited;

      if not (List.is_empty ps) then
        failwith
          "Malformed program, should not contain phi instructions";

      (* Generate label *)
      gen_label out l;

      (* Generate instructions *)
      gen_insts out is;

      (* Generate jump instruction *)
      match Lazy.force_val j with
      | IRPreg.CondJump (c, r, v, b1, b2) ->
        gen_condjump out c r v b1 b2;
        gen_irpreg_program_aux out b1;
        gen_irpreg_program_aux out b2;

      | IRPreg.Jump b ->
        gen_jump out b;
        gen_irpreg_program_aux out b;

      | IRPreg.Ret r ->
        gen_ret out r
    )
  in
  gen_section out ".text";
  gen_start out program;
  gen_irpreg_program_aux out program
;;
\end{lstlisting}

At line 14 we generate the label using the \texttt{gen\_label} function, then, at line 17 we generate the body of instructions of the basic block, and finally, at line 20, we generate the jump instruction. We continue with the recursion until we either reach a \texttt{Ret} instruction or we encounter an already visited block.

A subtle detail, when translating from \gls{jair} to x86-64 assembly, is that our language is a three-address code representation, while x86-64 assembly only allows instructions to have a maximum of two arguments.
To handle cases where the operands of an instruction are greater than two we introduce the following function:

\begin{lstlisting}[style=OCaml]
let gen_3ac_2ac out opcode r r' v =
  if r = r' then
    gen_bininst out opcode  (string_of_reg r) (string_of_val v)
  else
    match v with

    (* r(0) <- r(1) / i(100) *)
    | IRPreg.Imm x ->
      gen_bininst out MOV     (string_of_reg r) (string_of_reg r');
      gen_bininst out opcode  (string_of_reg r) (string_of_int x)

    (* r(0) <- r(1) / r(2) *)
    | IRPreg.Reg r'' ->
      gen_bininst out MOV     (string_of_reg tmp) (string_of_reg r');
      gen_bininst out opcode  (string_of_reg tmp) (string_of_reg r'');
      gen_bininst out MOV     (string_of_reg r)   (string_of_reg tmp);

    (* r(0) <- r(1) / p(100) *)
    | IRPreg.Ptr p ->
      gen_bininst out MOV     (string_of_reg r) (string_of_reg r');
      gen_bininst out opcode  (string_of_reg r) (string_of_int p)
;;
\end{lstlisting}

We handle three different cases, depending on the last operand:
\begin{itemize}
  \item If the third operand is an immediate value, we move the second register into the destination, and finally we perform the operation with the immediate value;
  \item If the third operand is another register, we move the second register into the temporary register, we perform the operation with the third register, and finally we move the temporary register into the destination register;
  \item If the third operand is a pointer, the behavior is the same as in the first point;
\end{itemize}
Now, because the temporary register, defined by \texttt{tmp}, is used only in the blocks introduced by the SSA destruction phase, it is safe to use it inside \texttt{Normal} blocks, since, during the coloring, we exclude \texttt{tmp} from the list of colors.

To show an example of a program translation, we take the program defined in \Cref{fig:example-jair} and compile it. The result is shown in \Cref{fig:example-nasm}.

\begin{figure}[ht]
\begin{lstlisting}[style=NASM]
section .text
global _start
_start:
	jmp	L1
L1:
	mov	rdx,	0
	mov	rcx,	1
	mov	rbx,	12
	mov	rax,	rdx
	mov	rdx,	rcx
	mov	rcx,	rax
	jmp	L2
L2:
	mov	rax,	rdx
	add	rax,	rcx
	mov	rcx,	rax
	sub	rbx,	1
	cmp	rbx,	1
	je	L2.1
	jmp	L2.2
L2.1:
	jmp	L3
L3:
	mov	rax,	60
	mov	rdi,	rcx
	syscall
L2.2:
	mov	rax,	rdx
	mov	rdx,	rcx
	mov	rcx,	rax
	jmp	L2
\end{lstlisting}
\caption{Assembly translation of the Fibonacci function}
\label{fig:example-nasm}
\end{figure}