% Errors

\chapter{Extraction}
\label{cha:extraction}

In this chapter, we put together the different components of our \gls{ra} pipeline. In \Cref{sec:pipeline}, we define the \texttt{regassign} function that does exactly that. In \Cref{sec:codegen}, we discuss the generation of \gls{nasm} code. Finally, in \Cref{sec:examples}, we present some examples of translated \gls{jair} programs.

\section{Composing the Pipeline}
\label{sec:pipeline}

After defining all the steps of our register assignment pipeline, we extract them into OCaml.
In particular, we extract the following functions:
\begin{itemize}
  \item \texttt{Vm.run} to interpret our programs;
  \item \texttt{LivenessInfo.analyze\_program} to obtain the liveness information of the program;
  \item \texttt{InterfGraph.get\_ig} to convert the liveness information into an interference graph;
  \item \texttt{Peo.eliminate} to obtain the \gls{peo};
  \item \texttt{Color.get\_coloring} to construct the coloring map;
  \item \texttt{Color.color\_program} to convert each virtual register of the program into its assigned physical register;
  \item \texttt{Destruct.ssa\_destruct} to translate the $\phi$-instructions;
\end{itemize}

After extraction, we compose the various phases of our \gls{ra} pipeline into the \texttt{regassign} function:

\begin{lstlisting}[style=OCaml]
let regassign irvreg_program =
  let (programinfo, _) = analyze_program irvreg_program fuel_analyze in
  let interfgraph = get_ig programinfo in
  let peo = eliminate interfgraph in
  let coloring =
    match get_coloring peo interfgraph with
    | Some c -> c
    | None -> failwith "Not enough registers to complete assignment"
  in
  let irpreg_program = color_program coloring irvreg_program in
  ssa_destruct fuel_destruct irpreg_program
;;
\end{lstlisting}

This function accepts a program written in \gls{jair} and returns an equivalent program without $\phi$-instructions and using only x86-64 registers.
Note that, at line 8, we panic if we are not able to perform the coloring due to a high number of variables, this check is mandatory, as we do not perform spilling.

\section{Code Generation}
\label{sec:codegen}

After obtaining the converted data structure, the generation of assembly code is straightforward. We simply perform a depth-first visit on the \gls{cfg} and generate the assembly code for each basic block.
In our implementation, we chose to emit assembly with the \gls{nasm} syntax.

The following OCaml function does exactly so, writing the assembly code of a program into the specified output channel:

\begin{lstlisting}[style=OCaml]
let gen_irpreg_program out program =
  let visited = ref LblSet.empty in
  let rec gen_irpreg_program_aux out program =
    let IRPreg.Block (l, ps, is, j) = Lazy.force_val program in

    if not (LblSet.mem l !visited) then (
      visited := LblSet.add l !visited;

      if not (List.is_empty ps) then
        failwith
          "Malformed program, should not contain phi instructions";

      (* Generate label *)
      gen_label out l;

      (* Generate instructions *)
      gen_insts out is;

      (* Generate jump instruction *)
      match Lazy.force_val j with
      | IRPreg.CondJump (c, r, v, b1, b2) ->
        gen_condjump out c r v b1 b2;
        gen_irpreg_program_aux out b1;
        gen_irpreg_program_aux out b2;

      | IRPreg.Jump b ->
        gen_jump out b;
        gen_irpreg_program_aux out b;

      | IRPreg.Ret r ->
        gen_ret out r
    )
  in
  gen_section out ".text";
  gen_start out program;
  gen_irpreg_program_aux out program
;;
\end{lstlisting}

At line 14 we generate the label of the current block, using the \texttt{gen\_label} function. Then, at line 17 we generate the body of instructions of the basic block. Finally, at line 20, we generate the jump instruction. We continue with the recursion until we either reach a \texttt{Ret} instruction or we encounter an already visited block.

It is important to note, that \gls{jair} is a three-address code representation. While \gls{nasm} only allows instructions that have a maximum of two arguments. Because of that, translation of binary operations must be treated specially.
Given physical registers \texttt A \texttt B, and \texttt C, we identify all the possible cases of a binary, non-commutative operation:
\begin{lstlisting}[style=Rocq]
Definition i1 : inst := r(A) <- r(A) / r(A).
Definition i2 : inst := r(A) <- r(A) / r(B).
Definition i3 : inst := r(A) <- r(B) / r(A).
Definition i4 : inst := r(A) <- r(B) / r(B).
Definition i5 : inst := r(A) <- r(B) / r(C).
\end{lstlisting}

% \begin{enumerate}
%   \item \label{itm:i1} \lstinline{Definition i1 : inst := r(A) <- r(A) / r(A).}
%   \item \label{itm:i2} \lstinline{Definition i2 : inst := r(A) <- r(A) / r(B).}
%   \item \label{itm:i3} \lstinline{Definition i3 : inst := r(A) <- r(B) / r(A).}
%   \item \label{itm:i4} \lstinline{Definition i4 : inst := r(A) <- r(B) / r(B).}
%   \item \label{itm:i5} \lstinline{Definition i5 : inst := r(A) <- r(B) / r(C).}
% \end{enumerate}

\lstset{style=NASM}
In instruction \texttt{i1}, a \gls{nasm} translation is straightforward, we can, in fact, use \lstinline{idiv A A}. The same goes for \texttt{i2}, where we can use \lstinline{idiv A B}. However, in \texttt{i3}, we see that no singular \gls{nasm} instruction can perform the intended operation. In particular, if we translate the operation as \lstinline{idiv A B}, we obtain an incorrect result, as division is non-commutative. Whereas, if we translate it as \lstinline{idiv B A}, the result will be written to register \texttt B, which is not the intended destination. For this case we use the following approach:

\begin{lstlisting}[style=NASM]
mov tmp A
mov A B
idiv A tmp
\end{lstlisting}

At line 1, we move the value of \texttt A into a temporary. At line 2, we move the content of \texttt B into the target register \texttt A. Finally, at line 3, we divide \texttt A (containing the initial value of \texttt B) by \texttt{tmp} (containing the initial value of \texttt A).
It is important to note that register \texttt{tmp} is only used inside \texttt{Point1} and \texttt{Point2} blocks. Because of that, there is no risk of colliding with a previous usage of that register.

Now, even \texttt{i4} requires multiple \gls{nasm} instructions, as a literal translation would still overwrite the content of \texttt B. We solve it this way, with two instruction:

\begin{lstlisting}
mov A B
idiv A B
\end{lstlisting}

The same goes for \texttt{i5}, which can be translated into:

\begin{lstlisting}
mov A B
idiv A C
\end{lstlisting}

In the previous examples, we used the division operator, which is binary and non-commutative. This is arguably the hardest case we could encounter, as for unary operators and load/store instructions, \gls{nasm} translation is straightforward.
We introduce a function specifically for handling the cases we just discussed:

\begin{lstlisting}[style=OCaml]
let gen_3ac_2ac out opcode r r' v =
  if r = r' then
    gen_bininst out opcode  (string_of_reg r) (string_of_val v)
  else
    match v with

    (* r(0) <- r(1) / i(100) *)
    | IRPreg.Imm x ->
      gen_bininst out MOV     (string_of_reg r) (string_of_reg r');
      gen_bininst out opcode  (string_of_reg r) (string_of_int x)

    (* r(0) <- r(1) / r(2) *)
    | IRPreg.Reg r'' ->
      if r = r'' then (
        gen_bininst out MOV     (string_of_reg tmp) (string_of_reg r');
        gen_bininst out opcode  (string_of_reg tmp) (string_of_reg r'');
        gen_bininst out MOV     (string_of_reg r)   (string_of_reg tmp)
      ) else (
        gen_bininst out MOV     (string_of_reg r) (string_of_reg r');
        gen_bininst out opcode  (string_of_reg r) (string_of_reg r'')
      )

    (* r(0) <- r(1) / p(100) *)
    | IRPreg.Ptr p ->
      gen_bininst out MOV     (string_of_reg r) (string_of_reg r');
      gen_bininst out opcode  (string_of_reg r) (string_of_int p)
;;
\end{lstlisting}

At line 2, we handle cases where the destination register is the same as the first operand. This translates directly into a \gls{nasm} instruction. At line 8, we handle the case where the second operand is an integer immediate value. Here, we just need to move the first operand into the destination register, and then perform the operation. At line 14, we implement the cases we discussed previously. Expect for \texttt{i1} and \texttt{i2}, which are handled together with the other instructions at line 2. Finally, at line 24, we handle the case where the second operand is a pointer. Since pointers are also immediate values, we emit the same instructions as the first branch of the match.

\section{Examples}
\label{sec:examples}

Here we present some \gls{jair} programs and their compiled version. We also include their return values, respectively on the \gls{jair} interpreter, and on x86-64 hardware.

% /-----------+
% | EXAMPLE 1 |
% +-----------/
\begin{figure}[ht]
\begin{minipage}{0.68\linewidth}
\centering
\begin{lstlisting}[style=Rocq]
Definition b3 : block :=
  Block (Normal 3) [] [] (ret r(5)).

CoFixpoint b2 : block :=
  Block (Normal 2) [
    (* Iterator *)
    r(2) <- phi [(0, Normal 1); (4, Normal 2)];
    (* Accumulator *)
    r(3) <- phi [(1, Normal 1); (5, Normal 2)]
  ] [
    r(4) <- r(2) - i(1);
    r(5) <- r(3) * r(4)
  ] (
    if r(4) <= i(1)
    then b3
    else b2
  ).

Definition b1 : block :=
  Block (Normal 1) [] [
    r(0) <- i(5);   (* Iterator *)
    r(1) <- r(0)    (* Accumulator *)
  ] (Jump b2).
\end{lstlisting}
\end{minipage}
\hfill
\begin{minipage}{0.28\linewidth}
\centering
\begin{lstlisting}[style=NASM]
section .text
global _start
_start:
  jmp L1
L1:
  mov rcx,  5
  mov rbx,  rcx
  jmp L2
L2:
  sub rcx,  1
  imul  rbx,  rcx
  cmp rcx,  1
  jle L2.1
  jmp L2.2
L2.1:
  jmp L3
L3:
  mov rax,  60
  mov rdi,  rbx
  syscall
L2.2:
  jmp L2
\end{lstlisting}
\end{minipage}
\caption{Translation of a program that computes the factorial of 5, both the \gls{jair} program and the \gls{nasm} program return 120.}
\label{fig:ex1}
\end{figure}

% /-----------+
% | EXAMPLE 2 |
% +-----------/
\begin{figure}[ht]
\begin{minipage}{0.68\linewidth}
\centering
\begin{lstlisting}[style=Rocq]
Definition b3 :=
  Block (Normal 3) [] []
    (ret r(3)).

CoFixpoint b2 :=
  Block (Normal 2) [
    r(2) <- phi [(0, Normal 1); (3, Normal 2)]
  ] [
    r(3) <- r(2) + i(3)
  ] (
    if r(3) < r(1)
    then b2
    else b3
  ).

Definition b1 :=
  Block (Normal 1) [] [
    r(0) <- i(0);
    r(1) <- i(20)
  ] (Jump b2).
\end{lstlisting}
\end{minipage}
\hfill
\begin{minipage}{0.28\linewidth}
\centering
\begin{lstlisting}[style=NASM]
section .text
global _start
_start:
  jmp  L1
L1:
  mov rbx,  0
  mov rcx,  20
  jmp L2
L2:
  add rbx,  3
  cmp rbx,  rcx
  jl  L2.1
  jmp L2.2
L2.1:
  jmp L2
L2.2:
  jmp L3
L3:
  mov rax,  60
  mov rdi,  rbx
  syscall
\end{lstlisting}
\end{minipage}
\caption{Translation of a program that increments a counter up to 21, both the \gls{jair} program and the \gls{nasm} program return 21.}
\end{figure}

% /-----------+
% | EXAMPLE 3 |
% +-----------/
\begin{figure}[ht]
\begin{minipage}{0.68\linewidth}
\centering
\begin{lstlisting}[style=Rocq]
Definition b3 : block :=
  Block (Normal 3) [] [] (ret r(6)).

CoFixpoint b2 : block :=
  Block (Normal 2) [
    r(3) <-
      phi [(0, Normal 1); (4, Normal 2)];
    r(4) <-
      phi [(1, Normal 1); (6, Normal 2)];
    r(5) <-
      phi [(2, Normal 1); (7, Normal 2)]
  ] [
    r(6) <- r(4) + r(3);  (* New first temp *)
    r(7) <- r(5) - i(1)   (* New iterator *)
  ] (
    if r(7) = i(1)
    then b3
    else b2
  ).

Definition b1 : block :=
  Block (Normal 1) [] [
    r(0) <- i(0);  (* Second temp *)
    r(1) <- i(1);  (* First temp *)
    r(2) <- i(12)  (* Iterator*)
  ] (Jump b2).
\end{lstlisting}
\end{minipage}
\hfill
\begin{minipage}{0.28\linewidth}
\centering
\begin{lstlisting}[style=NASM]
section .text
global _start
_start:
  jmp L1
L1:
  mov rdx,  0
  mov rcx,  1
  mov rbx,  12
  mov rax,  rdx
  mov rdx,  rcx
  mov rcx,  rax
  jmp L2
L2:
  mov rax,  rdx
  add rax,  rcx
  mov rcx,  rax
  sub rbx,  1
  cmp rbx,  1
  je  L2.1
  jmp L2.2
L2.1:
  jmp L3
L3:
  mov rax,  60
  mov rdi,  rcx
  syscall
L2.2:
  mov rax,  rdx
  mov rdx,  rcx
  mov rcx,  rax
  jmp L2
\end{lstlisting}
\end{minipage}
\caption{Translation of the Fibonacci of 12, both the \gls{jair} program and the \gls{nasm} program return 144.}
\end{figure}