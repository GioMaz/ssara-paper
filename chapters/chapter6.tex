% Errors

\chapter{Verification}
\label{cha:verification}

In this chapter we present the reasoning behind the verification of the core components of our register assignment pipeline, namely, the verification of the termination and the partial correctness of the function for obtaining the \gls{peo}, defined in \Cref{subsec:peo}. Here, we report the same theorems and the respective proofs that can be found in \texttt{Peo.v} in their mechanized version.

\section{Termination of \texttt{eliminate}}

As mentioned in \Cref{subsec:funterm}, to allow a function definition, Rocq requires a proof of its termination, for functions that use structural recursion the proof of termination is inferred automatically, for other functions instead, an explicit proof of termination is required. This is the case for \texttt{eliminate}.

Take the \texttt{eliminate} function defined in \Cref{subsec:peo}, we recursively check the result of \texttt{eliminate\_step}, and if the result is \texttt{Some} we continue with the recursion, otherwise we return an empty list. Proving termination for this function consists in proving that, after an iteration, the size of the graph decreases at least by one. Being the size a natural number, at some point it will reach a fixed point, that is, the \texttt{O} constructor.

\begin{theorem}[Termination of \texttt{eliminate}]
    Let graph $G = (V, E)$, after performing one elimination step, we obtain $G' = (V', E')$ such that $V' \subsetneq V$.
\end{theorem}
\begin{proof}
    After one iteration of the algorithm we encounter two possible cases, if no simplicial node is found, we terminate the function immediately. Otherwise, we found a simplicial node $u$, along with the new graph $G' = (V', E')$, in this case $V' = V \setminus \{ u \}$. Finally, since $u \in V$ we can say that $V' \subsetneq V$.
\end{proof}

\section{First Invariant of \texttt{eliminate\_step}}

We start by providing a constructive definition of the $\mathrm{Simplicial}$ predicate. After that, we introduce the soundness lemma for the \texttt{is\_simplicialb} function, defined in \Cref{subsec:peo}, with respect to $\mathrm{Simplicial}$.

Before defining the predicate, let us introduce some notation.

\begin{definition}[Node Addition]
    Let $G = (V, E)$ and $u$ be a node, we define the addition of the node $u$ as:
    \[
        G + u := (V \cup \{ u \}, E)
    \]
\end{definition}

\begin{definition}[Node Removal]
    Let $G = (V, E)$ and $u$ be a node, we define the removal of the node $u$ as:
    \[
        G - u := (V \setminus \{ u \}, \{ \{ x, y \} \mid x \neq u \land y \neq u \})
    \]
\end{definition}

\begin{definition}[Edge Addition]
    Let $G = (V, E)$ and $\{ u, v \}$ be an edge, we define the addition of the edge $\{ u, v \}$ as:
    \[
        G + \{ u, v \} := (V \cup \{ u, v \}, E \cup \{ \{ u, v \} \})
    \]
\end{definition}

\begin{definition}[Universal Vertex Over a Subgraph]
    Let $G = (V, E)$ and $H$ be a subgraph of $G$ with vertex set $V(H)$, we define the graph with a universal vertex $u$ over $H$ as:
    \[
        \addvertex{G}{H}{u} := (\{ V \cup \{ u \}, E \cup \{ \{ u, v \} \;|\; v \in V(H) \} \})
    \]
\end{definition}

\begin{definition}[Simplicial]
Let $G = (V, E)$, the predicate $\mathrm{Simplicial}(u, G)$ on the nodes of a graph is defined inductively by the following rules:
\begin{mathpar}
    \inferrule*[Right=SimplicialSingleton]
        {\neg V(u)}
        {\mathrm{Simplicial}(u, G + u)}
    \\
    \inferrule*[Right=SimplicialNode]
        {\mathrm{Simplicial}(u, G) \\ u \neq v}
        {\mathrm{Simplicial}(u, G + v )}
    \\
    \inferrule*[Right=SimplicialEdge]
        {\mathrm{Simplicial}(u, G) \\ u \neq v' \\ u \neq v''}
        {\mathrm{Simplicial}(u, G + \{ v', v'' \} )}
    \\
    \inferrule*[Right=SimplicialNeighbor]
        {\mathrm{Simplicial}(u, G)}
        {\mathrm{Simplicial}(u, \addvertex{G}{N(u)}{v})}
\end{mathpar}
\end{definition}

\begin{lemma}[Soundness of \texttt{is\_simplicialb}]\label{lem:sbsound}
    Let $G = (V, E)$ and $u \in V$ such that \texttt{is\_simplicialb} returns \texttt{true}, then $\mathrm{Simplicial}(u, G)$ holds.
\end{lemma}

\begin{theorem}[First Invariant of \texttt{eliminate\_step}]\label{thm:inv-elim-1}
    Let $G = (V, E)$, if \texttt{eliminate\_step} returns node $u$, then $\mathrm{Simplicial}(u, G)$.
\end{theorem}
\begin{proof}
    Assume that \texttt{eliminate\_next} returns node $u$, this means that the boolean property \texttt{is\_simplicialb} is \texttt{true} for that node. It follows immediately from \Cref{lem:sbsound} that $\mathrm{Simplicial}(u, G)$.
\end{proof}

\Cref{lem:sbsound} and \Cref{thm:inv-elim-1} are rather important, as they prove that the nodes inside a \gls{peo} are simplicial with respect to the graph where they are extracted.

\section{Second Invariant of \texttt{eliminate\_step}}

We introduce an inductive definition of the $\mathrm{Chordal}$ relation.

\begin{definition}[Chordal]\label{def:chordal2}
Let $G = (V, E)$, the predicate $\mathrm{Chordal}_2(G)$ is defined inductively by the following rules:
\begin{mathpar}
    \inferrule*[Right=ChordalEmpty]
        {}
        {\mathrm{Chordal}_2(\emptyset, \emptyset)}
    \\
    \inferrule*[Right=ChordalStep]
        {\exists u (\mathrm{Simplicial}(u, G) \land \mathrm{Chordal}_2(G - u))}
        {\mathrm{Chordal}_2(G)}
\end{mathpar}
\end{definition}

We can show that \Cref{def:chordal1}, and \Cref{def:chordal2} are equivalent. From now on we will call these definitions respectively $\mathrm{Chordal}_1$ and $\mathrm{Chordal}_2$. For the purpose of the subsequent proofs, we only need one side of the implication.

\begin{lemma}[Implication From Chordal$_1$ to Chordal$_2$]\label{lem:chordal12}
    Let $G = (V, E)$, if $\mathrm{Chordal}_1(G)$ then $\mathrm{Chordal}_2(G)$.
\end{lemma}

\begin{proof}
    Because of Chordal$_1(G)$, there exists a permutation $u_1, u_2, \dots, u_n$ which is a \gls{peo} . We proceed by induction over the size of the \gls{peo}.

    \medskip

    \textbf{Base case:}
    If the \gls{peo} is empty, this means that $G$ is the empty graph, because of the \textsc{ChordalEmpty} rule, Chordal$_2(G)$.

    \medskip

    \textbf{Inductive hypothesis:}
    Assume that $u_n, u_{n-1}, \dots, u_2, u_1$ being a \gls{peo}, implies Chordal$_2(G)$. Now, we add the node $u_{n+1}$ into the \gls{peo}, by either adding it as a singleton or with an edge, obtaining $G'$. We must prove that, if $u_{n+1}, u_n, u_{n-1}, \dots, u_2, u_1$ is a valid \gls{peo}, then Chordal$_2(G')$ holds.

    \medskip

    \textbf{Inductive step:}
    Let's take the case where $G' = G + u_{n+1}$. Assume $u_{n+1}, u_{n}, u_{n-1}, \dots, u_2, u_1$ is a valid \gls{peo}. Then, if we remove $u_{n+1}$ from the graph, because it is a singleton, it doesn't interfere with the rest of the nodes. We then know that $u_n, u_{n-1}, \dots, u_2, u_1$ is a \gls{peo}, and because of the inductive hypothesis, we know that Chordal$_2(G)$. Now, since $u_{n+1}$ is not connected to the rest of the graph, Simplicial$(u, G')$ is the case. Also, because $G = G' - u_{n+1}$, we can say that Chordal$_2(G' - u_{n+1})$. We then apply the \textsc{ChordalStep} rule, obtaining Chordal$_2(G')$.
    Now, let's take into consideration the case where $G' = G + \{ u_{n+1}, v \}$. In particular, the case where $v \in E$ and $u_{n+1} \not \in E$. In this case, $u_{n+1}$ is not a singleton. However, it has only one neighbor, namely $v$, and so it's still simplicial. Because of that we can repeat the same procedure as before.
\end{proof}

Now, we introduce the completeness theorem for \texttt{is\_simplicialb} with respect to $\mathrm{Simplicial}$.

\begin{lemma}[Completeness of \texttt{is\_simplicialb}]\label{lem:sbcomp}
Let $G = (V, E)$ and $u \in V$ such that $\mathrm{Simplicial}(u, G)$ holds, then \texttt{is\_simplicialb} returns \texttt{true}.
\end{lemma}

\begin{theorem}[Second Invariant of \texttt{eliminate\_step}]\label{cor:inv-elim-2}
    Let $G = (V, E)$, if $\mathrm{Chordal}_1(G)$ and \texttt{eliminate\_step} does not return any node, then $G = (\emptyset, \emptyset)$.
\end{theorem}
\begin{proof}
    Assume Chordal$_1(G)$ and that \texttt{eliminate\_next} does not return any node. By \Cref{lem:chordal12}, Chordal$_2(G)$. As stated in \Cref{def:chordal2}, either $G$ is the empty graph, or there exists $u$ such that $\mathrm{Simplicial(u, G)}$ and $\mathrm{Chordal}_2(G - u)$. In the second case, \Cref{lem:sbcomp} lets us derive that \texttt{is\_simplicialb} returns \texttt{true} for $u$. However, since \texttt{eliminate\_step} did not return any value, this contradicts the assumption. Because of that, the only possible case is the one where $G = (\emptyset, \emptyset)$.
\end{proof}