% Errors

\chapter{Verification}
\label{cha:verification}

In this chapter we present the reasoning behind the verification of the core components of our register assignment pipeline, namely, the verification of the termination and the partial correctness of the function for obtaining the \gls{peo}, defined in \Cref{subsec:peo}. Here, we report the same theorems and the respective proofs that can be found in \texttt{Peo.v} in their mechanized version.

\section{Termination of \texttt{eliminate}}

As mentioned in \Cref{subsec:funterm}, to allow a function definition, Rocq requires a proof of its termination, for functions that use structural recursion the proof of termination is inferred automatically, for other functions instead, an explicit proof of termination is required. This is the case for \texttt{eliminate}.

Take the \texttt{eliminate} function defined in \Cref{subsec:peo}, we recursively check the result of \texttt{eliminate\_step}. If the result is \texttt{Some} we continue with the recursion, otherwise, we return an empty list. Proving termination for this function consists in proving that, after an iteration, the size of the new graph is strictly lower than the size of the original graph. Being the size a natural number, at some point it will reach a base case, that is, the \texttt{O} constructor.

\begin{theorem}[Termination of \texttt{eliminate}]\label{thm:term-elim}
    Let graph $G = (V, E)$, after performing one elimination step, we obtain $G' = (V', E')$ such that $V' \subsetneq V$. \Coqed
\end{theorem}
\begin{proof}
    After one iteration of the algorithm we encounter two possible cases, if no simplicial node is found, we terminate the function immediately. Otherwise, we found a simplicial node $u$, along with the new graph $G' = (V', E')$, in this case $V' = V \setminus \{ u \}$. Finally, since $u \in V$ we can say that $V' \subsetneq V$.
\end{proof}

\section{First Invariant of \texttt{eliminate\_step}}

We start by providing a constructive definition of the $\mathrm{Simplicial}$ predicate. After that, we introduce the soundness lemma for the \texttt{is\_simplicialb} function, defined in \Cref{subsec:peo}, with respect to $\mathrm{Simplicial}$.

Before defining the predicate, let us introduce some notation.

\begin{definition}[Node Addition]
    Let $G = (V, E)$ and $u$ be a node, we define the addition of the node $u$ as:
    \[
        G + u := (V \cup \{ u \}, E)
    \]
\end{definition}

\begin{definition}[Node Removal]
    Let $G = (V, E)$ and $u$ be a node, we define the removal of the node $u$ as:
    \[
        G - u := (V \setminus \{ u \}, \{ \{ x, y \} \mid x \neq u \land y \neq u \})
    \]
\end{definition}

\begin{definition}[Edge Addition]
    Let $G = (V, E)$ and $\{ u, v \}$ be an edge, we define the addition of the edge $\{ u, v \}$ as:
    \[
        G + \{ u, v \} := (V \cup \{ u, v \}, E \cup \{ \{ u, v \} \})
    \]
\end{definition}

\begin{definition}[Universal Vertex Over a Subgraph]
    Let $G = (V, E)$ and $H$ be a subgraph of $G$ with vertex set $V(H)$, we define the graph with a universal vertex $u$ over $H$ as:
    \[
        \addvertex{G}{H}{u} := (\{ V \cup \{ u \}, E \cup \{ \{ u, v \} \;|\; v \in V(H) \} \})
    \]
\end{definition}

\begin{definition}[Simplicial]
Let $G = (V, E)$, the predicate $\mathrm{Simplicial}(u, G)$ on the nodes of a graph is defined inductively by the following rules:
\begin{mathpar}
    \inferrule*[Right=SimplicialSingleton]
        {u \not \in V}
        {\mathrm{Simplicial}(u, G + u)}
        % {G + u \vdash u : \text{Simplicial}}
    \\
    \inferrule*[Right=SimplicialNode]
        {\mathrm{Simplicial}(u, G) \\ u \neq v}
        {\mathrm{Simplicial}(u, G + v )}
        % {G + v \vdash u : \text{Simplicial}}
    \\
    \inferrule*[Right=SimplicialEdge]
        {\mathrm{Simplicial}(u, G) \\ u \neq v' \\ u \neq v''}
        {\mathrm{Simplicial}(u, G + \{ v', v'' \} )}
    \\
    \inferrule*[Right=SimplicialNeighbor]
        {\mathrm{Simplicial}(u, G)}
        {\mathrm{Simplicial}(u, \addvertex{G}{N(u)}{v})}
\end{mathpar}
\end{definition}

\begin{lemma}[Soundness of \texttt{is\_simplicialb}]\label{lem:sbsound}
    Let $G = (V, E)$ and $u \in V$ such that \texttt{is\_simplicialb} returns \texttt{true}, then $\mathrm{Simplicial}(u, G)$ holds. \Coqed
\end{lemma}

\begin{theorem}[First Invariant of \texttt{eliminate\_step}]\label{thm:inv-elim-1}
    Let $G = (V, E)$, if \texttt{eliminate\_step} returns node $u$, then $\mathrm{Simplicial}(u, G)$. \Coqed
\end{theorem}
\begin{proof}
    Assume that \texttt{eliminate\_next} returns node $u$, this means that the boolean property \texttt{is\_simplicialb} is \texttt{true} for that node. It follows immediately from \Cref{lem:sbsound} that $\mathrm{Simplicial}(u, G)$.
\end{proof}

\Cref{lem:sbsound} and \Cref{thm:inv-elim-1} are rather important, as they prove that the nodes inside a \gls{peo} are simplicial with respect to the graph where they are extracted.

\section{Second Invariant of \texttt{eliminate\_step}}

We introduce an inductive definition of chordality with the $\chordalind{\cdot}$ predicate.

\begin{definition}[$\chordalind{\cdot}$]\label{def:chordal2}
Let $G = (V, E)$, the predicate $\chordalind{G}$ is defined inductively by the following rules:

% \begin{center}
%     \judgboxb{\chordalone{G}}{ABCDcidsojcccc \\ icjsojc c\\ cjoisjc}
% \end{center}

\begin{mathpar}
    \inferrule*[Right=ChordalEmpty]
        {\empty}
        {\chordalind{(\emptyset, \emptyset)}}
    \\
    \inferrule*[Right=ChordalStep]
        {\exists u (\mathrm{Simplicial}(u, G) \land \chordalind{G - u})}
        {\chordalind{G}}
\end{mathpar}
\end{definition}

From now on, we will refer to chordality as \Cref{def:chordal1} $\chordalpeo{\cdot}$, to distinguish it from the $\chordalind{\cdot}$ predicate we just introduced. Intuitively, these two definitions are equivalent. In order to prove it, we start with the forward implication.

\begin{lemma}[Implication From $\chordalpeo{\cdot}$ to $\chordalind{\cdot}$]\label{lem:chordal12}
    Let $G = (V, E)$, if $\chordalpeo{G}$ then $\chordalind{G}$.
\end{lemma}

\begin{proof}
    Because of $\chordalpeo{G}$, there exists a permutation $u_1, u_2, \dots, u_n$ which is a \gls{peo}. We proceed by induction over the length of the \gls{peo}.

    \medskip

    \textbf{Base case:}
    If the \gls{peo} is empty, this means that $G$ is the empty graph. Now, because of the \textsc{ChordalEmpty} rule, $\chordalind{G}$ holds.

    \medskip

    \textbf{Inductive hypothesis:}
    Assume that, if $u_1, u_2, \dots, u_n$ is a \gls{peo}, then $\chordalind{G}$. Now, we add the simplicial node $u_{n+1}$ into the \gls{peo}, obtaining $G'$. We must prove that, if the sequence $u_1, u_2, \dots u_n, u_{n+1}$ is a \gls{peo}, then $\chordalind{G'}$ holds.

    \medskip

    \textbf{Inductive step:}
    A \gls{peo} is an ordering where each node is connected only to the nodes in previous positions. Because of that, if any node is simplicial in $G$, then it is also simplicial in $G' - u_{n+1}$. Because of that, if $\chordalpeo{G'}$ then $\chordalpeo{G' - u_{n+1}}$, consequently, $\chordalind{G}$. Now, once again, we add $u_{n+1}$ into the graph $G$, such that Simplicial$(u_{n+1}, G')$. Moreover, we know that $\chordalind{G}$, which can also be represented as $\chordalind{G' - u_{n+1}}$. The two propositions we just presented, are exactly the premises of \textsc{ChordalStep}. Because of that, $\chordalind{G'}$
\end{proof}

Now, we prove the backwards implication.

\begin{lemma}[Implication From $\chordalind{\cdot}$ to $\chordalpeo{\cdot}$]\label{lem:chordal21}
    Let $G = (V, E)$, if $\chordalind{}{G}$ then $\chordalpeo{}{G}$.
\end{lemma}
\begin{proof}
    We proceed by induction, this time, over the size of the graph.

    \textbf{Base case:}
    If $G$ is empty, there exists a \gls{peo}, which simply is an empty sequence.

    \medskip

    \textbf{Inductive hypothesis:}
    Assume that, if for all the graphs $G = (V, E)$ such that $|V| < n$, if $\chordalind{G}$, then there exists a \gls{peo} for $G$. We need to prove that for all the graphs $G' = (V', E')$ such that $|V'| < n+1$, if $\chordalind{G'}$, then there exists a \gls{peo} for $G'$.

    \medskip

    \textbf{Inductive step:}
    Because $\chordalind{G'}$, there exists a node $v$ such that this node is simplicial in $G'$ and $\chordalind{G' - v}$. Since $|V(G' - v)| < n$, we apply the inductive hypothesis, obtaining that $u_1, u_2, \dots, u_n$ is a valid \gls{peo} for $G' - v$. Because $v$ is simplicial in $G'$, we can create a \gls{peo} for $G'$ which is $u_1, u_2, \dots, u_n, v$.
\end{proof}

Now, we introduce the completeness theorem for \texttt{is\_simplicialb} with respect to $\mathrm{Simplicial}$.

\begin{lemma}[Completeness of \texttt{is\_simplicialb}]\label{lem:sbcomp}
Let $G = (V, E)$ and $u \in V$ such that $\mathrm{Simplicial}(u, G)$ holds, then \texttt{is\_simplicialb} returns \texttt{true}. \Coqed
\end{lemma}

\begin{theorem}[Second Invariant of \texttt{eliminate\_step}]\label{thm:inv-elim-2}
    Let $G = (V, E)$, if it is the case that $\chordalpeo{G}$, and \texttt{eliminate\_step} does not return any node, then $G = (\emptyset, \emptyset)$. \Coqed
\end{theorem}
\begin{proof}
    Assume $\chordalpeo{G}$ and that \texttt{eliminate\_next} does not return any node. By \Cref{lem:chordal12}, $\chordalind{G}$ holds. As stated in \Cref{def:chordal2}, either $G$ is the empty graph, or there exists $u$ such that $\mathrm{Simplicial(u, G)}$ and $\chordalind{G - u}$. In the second case, \Cref{lem:sbcomp} lets us derive that \texttt{is\_simplicialb} returns \texttt{true} for $u$. However, this contradicts the assumption, since \texttt{eliminate\_step} did not return any value. Because of that, the only possible case is the one where $G = (\emptyset, \emptyset)$.
\end{proof}

\section{Partial correctness of \texttt{eliminate\_step}}

After defining the two invariants for the \texttt{eliminate\_step} function, We prove its partial correctness.

\begin{theorem}[Partial Correctness of \texttt{eliminate\_step}]\label{thm:par-cor}
    Let $G = (V, E)$ such that $\chordalpeo{G}$, if \texttt{eliminate\_step} returns node $u$, then $\mathrm{Simplicial}(u, G)$, otherwise $G = (\emptyset, \emptyset)$. \Coqed
\end{theorem}

An additional result from graph theory is the following:

\begin{theorem}[Chordality Invariant After Node Removal]\label{thm:cho-inv}
    Given graph $G = (V, E)$, if $\chordalpeo{G}$, then $\chordalpeo{G - u}$ for an arbitrary $u$~\cite{golumbic2004algorithmic}.
\end{theorem}

We do not include a proof for \Cref{thm:cho-inv}. However, it is an important proposition as it states that, if an initial chordal graph is given, the \texttt{eliminate} function will ultimately return a \gls{peo} that contains every node of the graph.