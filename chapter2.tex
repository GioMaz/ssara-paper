\chapter{Background}
\label{cha:background}
In this chapter we will introduce the tools we will use during the course of our project, namely the Coq Proof Assistant among with some of its core features that make it suitable for a building register allocator. We will also introduce a specification of SSA form, the intermediate representation that we will use and finally some theoretical aspects about register allocation.

\section{The Coq Proof Assistant}
The implementation of the core of our register allocator will be performed in Coq. Coq is an interactive proof assistant that leverages a strongly typed functional programming language called Gallina for its computational part. 
During the course of the project will in fact focus more on the computational aspect of the implementation rather than proving the correctness of said algorithms.

\subsection{Function termination}
One feature of Coq that we will make extensive use of is the termination checking for functions. Using the \textbf{Fixpoint} keywork lets us insert a recursive function definition that pattern matches over inductive types.
During the evaluation of said definition Coq checks for the existence of a decreasing argument (between the arguments of the function) and does that for every branch of the match expression. If the previous condition is satisfied then this means that at some point we will reach the non-recursive constructor of the inductive type and terminate.
Intuitively there exist more complicated functions that do not leverage this mechanism meaning that Coq will not be able to prove their termination easily, to make this functions acceptable we can proceed in two different ways:
the first one is the easiest, it consists in forcing into the function signature a decreasing argument that the compiler can check, the argument is usually something like \textbf{(fuel : nat)} and the idea is that the algorithm iterates until \textbf{fuel} reaches zero. The problem with this solution is that we must know beforehand how many iterations we will need which is often a non-trivial task. If we fail to do so we could risk to terminate our function before the algorithm is actually completed ending up with an incorrect result.
The second way to ensure termination is to use the \textbf{Function} keyword and then provide Coq with a proof of the termination of the algorithm.
We will mostly use the first option assuming we know how many iterations we need for each algorithm.

\subsection{OCaml extraction}
\todo[inline]{extraction}

\section{Single Static Assigment}
\label{sec:ssa}
For this project we use the Single Static Assignment (or SSA) form which is a set of intermediate languages that, among other benefits, drastically simplify the process of register allocation.
To qualify as SSA form an intermediate language must satisfy the following constraints:
\begin{itemize}
    \item Each variable (or virtual register) is defined exactly once;
    \item Each variable (or virtual register) usage is dominated by its definition;
\end{itemize}
We say that label $l_1$ dominates label $l_2$ iff for every path from the start of the control flow graph to $l_2$ we encounter $l_1$.

In practice to make an SSA form language usable we must include some additional features.

\subsection{Control Flow Graphs}
To match the expressiveness of programming languages we represent a program as a control flow graph (CFG) which is defined as the triple $(B, CF, \textbf{start})$ where $B$ is the set of basic blocks which are linear sequences of instructions that terminate with a branch instruction. Then we have $CF \subseteq B \times B$ which is the edge relation that represents the control flow. And finally $\textbf{start} \in B$ which is the starting point of the program.

\subsection{Phi Operations}
As a direct consequence of the single assignment policy we are forced to introduce the $\phi$-operation which is an instruction of the form $x \leftarrow \phi((y_1, b_1), (y_2, b_2), \dots, (y_n, b_n))$ where $y_1, y_2, \dots, y_n$ are variables live respectively at blocks $b_1, b_2, \dots, b_n$.
The phi operations are are always placed at the start of a basic block, here follows an example:
\[
\begin{pmatrix}
    x_1 \\ x_2 \\ \vdots \\ x_m
\end{pmatrix}
\leftarrow \phi
\begin{pmatrix}
    (y_{11}, b_1) & (y_{12}, b_2) & \dots & (y_{1n}, b_n) \\
    (y_{21}, b_1) & (y_{22}, b_2) & \dots & (y_{2n}, b_n) \\
    \vdots & \vdots & \ddots & \vdots \\
    (y_{m1}, b_1) & (y_{m2}, b_2) & \dots & (y_{mn}, b_n) \\
\end{pmatrix}
\]
Semantically, this section is executed during the jump from the predecessor block to the current block and its evaluation assigns the variable corresponding to the predecessor block $b_j$ to $x_i$.
Another important feature of phi semantics is that the phi instructions get executed in parallel, it is important to keep this in mind when translating phi operations into machine code.

\section{Register Allocation}
\label{sec:ra}

Register Allocation is the process of mapping the set of variables, which can be virtually infinite, to a set of finite physical registers. It is a known result that register allocation can be reduced to graph coloring. First we take the interference graph $G = (V, E)$ where $V$ is the set of variables and $E$ is the edge relation such that $(u, v) \in E$ iff there is a point in our program where $u$ and $v$ are both live, this intuitively implies that variables $u$ and $v$ cannot be assigned to the same physical register.
We can then use a graph coloring algorithm on the interference graph where each color represents a physical register and the result will be a valid solution for the register allocation problem.
Obtaining a $k$-coloring of an interference graph thus provides us with a valid register allocation for a machine with $k$ physical registers.

\subsection{Register Allocation in SSA form}
As we know graph coloring of a general graph is an $\mathcal{NP}$-complete problem, however programs in SSA form have an interference graph which is chordal and coloring a graph with this restriction can be solved in quadratic time.
