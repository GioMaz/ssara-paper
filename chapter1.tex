% \chapter{Introduction}
% \label{cha:intro}

% \section{Register Allocation}
% \label{sec:context}
% This document will outline some theoretical and practical aspects of the
% creation of a register allocator for a single static assignment (SSA)
% intermediate representation.
% For the implementation of the register allocator we will use the Coq proof
% assistant and, in particular, its functional language ``Gallina'' that provides
% built-in features for, among other functionalities, ensuring termination of the
% presented algorithms.

% \section{Complexity of the Register Allocation Problem}
% \label{sec:problem}
% However, as we will later prove, the register allocation procedure is an
% $\mathcal{NP}$-complete problem as any instance of it can be reduced to an
% instance of the graph coloring problem.

% \section{An heuristic algorithm}
% \label{sec:solution}
% We make the following contributions:
% \begin{itemize}
%     \item We define an intermediate representation in SSA form suited for
%     register allocation. The representation must be simple enough not to be a
%     burden on the frontend, but at the same be expressive enough to facilitate
%     the register allocation phase and avoid any unnecessary computation.
%     Along with the syntax for our IR, we define clear semantics by implementing
%     an interpreter.
%     \item We provide  insights on the implementation of the following steps of a
%     register allocator:
%     \begin{itemize}
%         \item Liveness analysis and creation of the interference graph
%         \item Register allocation (graph coloring)
%         \item SSA destruction
%     \end{itemize}
% \end{itemize}